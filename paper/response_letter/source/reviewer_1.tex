% chktex-file 1
% chktex-file 46
\section{Reviewer \#1}\label{reviewer_1}

Reviewer \#1 Evaluations:
Recommendation (Required): Return to author for minor revisions
Significant: The paper has some unclear or incomplete reasoning but will likely be a significant contribution with revision and clarification.
Supported: Mostly yes, but some further information and/or data are needed.
Referencing: Yes
Quality: Yes, it is well-written, logically organized, and the figures and tables are appropriate.
Data: Yes
Accurate Key Points: Yes

\subsection*{Overall comments}

Reviewer \#1 (Formal Review for Authors (shown to authors)):

\RC{} The study on water governance stages relies heavily on the integrated index IWGI at the basin scale, which offers significant advantages for decision-making support. Nonetheless, in large basins, the index may obscure crucial details and spatial heterogeneity. For instance, averaging the water stress of areas with and without water stress could lead to a misleading "no water stress" result. To improve the discussion on water governance transition, I recommend the authors link the three periods to well-known phenomena or policies, such as the occurrence of zero-flow in the Yellow River.

\AR{} We sincerely appreciate your constructive feedback and acknowledgment of our work. We concur that the IWGI index is designed to provide an overall understanding of the basin, and in this process, some spatial heterogeneity may indeed be overlooked.

\AR*{} We would like to clarify that our intention in averaging water stress across the basin is not to obscure specific areas of concern but to provide a comprehensive view that can strongly support decision-making. We recognize that in a large basin like the Yellow River, disparities in water stress exist. However, these disparities, where one area might experience high water stress while another experiences low, can influence water allocation and management strategies, thus are crucial to capture at a macro level.

\RC{} Additionally, the current separation of periods using the Pettitt method for change point detection could be improved or compared to other methods, considering its sensitivity to time series changes.

\AR{} Regarding the separation of periods using the Pettitt method for change point detection, we have taken your feedback into account. To address this, we have incorporated results from other change point detection techniques to demonstrate the robustness of our stage delineations, particularly in the context of the Yellow River basin. The updated results indicate a consistent pattern, reaffirming the credibility of our initial findings:

\begin{quote}
	The results from the Pettitt method are robust to the inclusion of other change point detection techniques.
\end{quote}

\subsection*{Specific comments}

Equation (5): Would adopting an area-weighted average be more appropriate in Equation (5)? This change could offer a more accurate representation of the data, especially when considering the variations in water stress across different areas within the basin.

Page 5: I guess WUi,j reflects water use (or water withdrawal) rather than water consumption.

Page 10, Line 265-266: It is crucial to rectify the statement claiming lower water stress during Period 1 (P1) as it contradicts the occurrence of zero-flow in the river five times between 1972 and 1978, with an average duration of 13 days per occurrence. Furthermore, since 2000, zero-flow has not reappeared, which could be accounted for in the analysis.

Page 11, Line 308: Could you elaborate on "manage natural water processes"?

Page 11, Line 313-314: For better comprehension, consider listing some of the environmental, economic, social, and policy challenges faced during the various water governance stages.

Page 11, Line 314-317: Please clarify the purpose of this sentence to ensure its relevance to the discussion.
