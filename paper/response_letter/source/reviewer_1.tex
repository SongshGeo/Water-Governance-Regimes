% chktex-file 1
% chktex-file 46
\section{Reviewer \#1}\label{reviewer_1}

Reviewer \#1 Evaluations:
Recommendation (Required): Return to author for minor revisions
Significant: The paper has some unclear or incomplete reasoning but will likely be a significant contribution with revision and clarification.
Supported: Mostly yes, but some further information and/or data are needed.
Referencing: Yes
Quality: Yes, it is well-written, logically organized, and the figures and tables are appropriate.
Data: Yes
Accurate Key Points: Yes

\subsection*{Overall comments}

Reviewer \#1 (Formal Review for Authors (shown to authors)):

\RC{} The study on water governance stages relies heavily on the integrated index IWGI at the basin scale, which offers significant advantages for decision-making support. Nonetheless, in large basins, the index may obscure crucial details and spatial heterogeneity. For instance, averaging the water stress of areas with and without water stress could lead to a misleading ``no water stress'' result. To improve the discussion on water governance transition, I recommend the authors link the three periods to well-known phenomena or policies, such as the occurrence of zero-flow in the Yellow River.

\AR{} We sincerely appreciate your constructive feedback and acknowledgment of our work. We concur that the IWGI index is designed to provide an overall understanding of the basin, and in this process, some spatial heterogeneity may indeed be overlooked.

\AR*{} We would like to clarify that our intention in averaging water stress across the basin is not to obscure specific areas of concern but to provide a comprehensive view that can strongly support decision-making. We recognize that in a large basin like the Yellow River, disparities in water stress exist. However, these disparities, where one area might experience high water stress while another experiences low, can influence water allocation and management strategies, thus are crucial to capture at a macro level.

\RC{} Additionally, the current separation of periods using the Pettitt method for change point detection could be improved or compared to other methods, considering its sensitivity to time series changes.

\AR{} Regarding the separation of periods using the Pettitt method for change point detection, we have taken your feedback into account. To address this, we have incorporated results from other change point detection techniques to demonstrate the robustness of our stage delineations, particularly in the context of the Yellow River basin. The updated results indicate a consistent pattern, reaffirming the credibility of our initial findings.

Specifically, we employed the Generalized ESD Test, Dynamic Programming, Binary Segmentation, and Bottom-Up Segmentation methods to identify the change points in our index. The Generalized ESD Test, which does not require the pre-specification of the number of change points, identified three significant points: 1975, 1978, and 2000. On the other hand, Dynamic Programming, Binary Segmentation, and Bottom-Up Segmentation, which require the pre-specification of the number of change points, consistently pointed to 1975 and 2000.

We observed that the first change point identified by the Generalized ESD Test is close to 1975 and 1978. After careful consideration, we opted for 1978 as the first change point to ensure that the total number of years across the three identified periods is more balanced. This decision aligns with our objective to provide a more comprehensive and equitable analysis of the different time frames, thereby enhancing the robustness and reliability of our findings.

\begin{quote}
	We have also employed a series of popular breakpoint detection methods to validate the results obtained through the Pettitt test. The names of these methods, their respective results, and the references to the pertinent literature can be found in the accompanying table S2. A consensus was observed among all methods regarding the identification of the second breakpoint, which consistently emerged around the year 2000. There was a minor variation of three years in the identification of the first breakpoint, pinpointed either in 1975 or 1978.

	% 我们综合考虑每个时间段的年份数量后,选择了较为居中的、识别出1977年作为第一个断点的 Pettitt 方法。但可以看得出,本研究所依赖的断点识别结果,对于算法选择的鲁棒性也足够强。
	After a thorough evaluation, we opted for the Pettitt method, which identified 1977 as the first breakpoint. This decision was influenced by our intent to ensure a balanced distribution of the number of years across each identified period. The consistency in the identification of breakpoints, especially the second one, across different methods underscores the robustness of our findings, attesting to their insensitivity to the choice of the breakpoint detection algorithm. This robustness reinforces the credibility and reliability of our analyses and conclusions, reflecting the methodological rigor embedded in our study.
\end{quote}

\AR*{} We believe that the consistency in the change points identified by different methods, despite their varied approaches and assumptions, underscores the robustness of our period delineations. We have included these additional analyses in the revised manuscript to offer a more comprehensive view and address any concerns regarding the sensitivity of our results to the change point detection method employed.

\AR*{} We hope this clarification and the additional analyses sufficiently address your concern, and we are open to further discussions to enhance the quality of our work.

\subsection*{Specific comments}

\RC{} Equation (5): Would adopting an area-weighted average be more appropriate in Equation (5)? This change could offer a more accurate representation of the data, especially when considering the variations in water stress across different areas within the basin.

\AR{} We are grateful for your insightful suggestion regarding the incorporation of area-weighted averages in Equation (5). In our current study, we have adhered strictly to the algorithm proposed by Qin et al., which does not employ any weighting scheme. Our objective is to maintain consistency and avoid introducing additional complexities that could potentially lead to confusion in the interpretation of our results.

\AR*{} However, we acknowledge the potential benefits of incorporating a weighting factor to account for the variations in water stress across different areas within the basin. In future studies, considering weights such as population or other socio-economic indicators might indeed be more appropriate. The underlying rationale would be that areas with higher populations facing water stress indicate a more severe shortage of water resources.

\AR*{} This could represent a significant enhancement to the SFV index, offering a more nuanced and comprehensive representation of water stress at the basin scale. We are genuinely appreciative of your suggestion and are considering integrating this improvement in our subsequent works focusing on the refinement of the SFV index.

\RC{} Page 5: I guess WUi,j reflects water use (or water withdrawal) rather than water consumption.

% 我们非常感谢你方对细节的关注和对我方术语的更正。我们已经复习了有问题的部分,并修改了术语,以确保我们的陈述准确和清晰。
\AR{} We greatly appreciate your attention to detail and the correction you provided regarding our terminology. We have reviewed the section in question and have amended the terminology to ensure precision and clarity in our presentation:

\begin{quote}
	First, for scarcity, $A_{i, j}$ is the total water use as a proportion of regional multi-year average runoff volume in year $j$ and region $i$ \dots
\end{quote}

\RC{} Page 10, Line 265\-266: It is crucial to rectify the statement claiming lower water stress during Period 1 (P1) as it contradicts the occurrence of zero-flow in the river five times between 1972 and 1978, with an average duration of 13 days per occurrence. Furthermore, since 2000, zero-flow has not reappeared, which could be accounted for in the analysis.

% 1. 首先感谢指出这个表述问题,这确实造成了误解。
% 2. 其次,第一阶段确实出现了断流,但正如下图所展示的,断流的天数和长度都还偏小。第一阶段确实相对来说已经是压力比较小的,但增大的太过迅速。
% 3. 正如我们在新增的不同断点识别方法所诊断出的第一阶段结束在 1975-1978之间,而第一次出现断流的时间是1973年。就我们研究的时间尺度来说,已经是比较接近的识别了。但数据限制,没有办法非常完全准确到第一阶段没有断流。
% 4. 总的来说,第一阶段相对其他时期压力并不算大,但压力增速极大,导致第一阶段末出现断流现象。我们已经删除了容易歧义的表述。
\AR{} We extend our sincere gratitude for highlighting the inconsistency in our description. We acknowledge that our previous phrasing may have led to misunderstandings. We have taken the liberty to revise our manuscript, eliminating ambiguous phrases to ensure clarity and accuracy in paper.

\AR*{} We recognize that zero-flow events occurred during the first stage. However, both the duration and frequency of these events were relatively minor (worst in 1990s). While the water stress was indeed lower during this period, the rapid increase in stress levels was a significant concern.

\AR*{} With the introduction of additional breakpoint detection methods in our revised manuscript, we observed that the first stage ends between 1975 and 1978, aligning closely with the onset of zero-flow events in 1973. Given the temporal scale of our study and the limitations inherent in the available data, this alignment is reasonably accurate.

\AR*{} In summation, while the water stress during the first stage was not as pronounced as in subsequent periods, the accelerated rate of increase in stress levels culminated in zero-flow events towards the end of this stage.

\RC{} Page 11, Line 308: Could you elaborate on ``manage natural water processes''?

\AR{} Thank you for bringing to our attention the ambiguity associated with the term ``natural water processes''. To enhance clarity and precision, we have revised this phrase to ``natural water cycle'' in our updated manuscript.

\RC{} Page 11, Line 313\-314: For better comprehension, consider listing some of the environmental, economic, social, and policy challenges faced during the various water governance stages.

\RC{} Page 11, Line 314\-317: Please clarify the purpose of this sentence to ensure its relevance to the discussion.
