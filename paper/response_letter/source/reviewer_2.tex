\section{Reviewer \#2}\label{reviewer_2}

\subsection*{Overall comments}
% 作者提出了IWGI,并将其应用于黄河流域的水治理历史,分析了流域的制度变迁。该方法具有很强的合理性和适应性,案例分析的结果为理解中国的水治理提供了有益的见解。然而,讨论并没有充分利用案例研究分析的结果。讨论结果的内容应该更具有支持性和可靠性。此外,讨论的叙事分析非常有趣,但如果这项研究的原创性之一是识别变化点,那么就需要更多地讨论导致这些变化点的触发因素。请参考我下面的审稿意见并进行修改,以提高本文的价值。
\RC{} The authors proposed the IWGI, applied the index to the history of water governance in the Yellow River basin, and analyzed the regime change in the basin. The methodology is very reasonable and adaptable, and the results of the case analysis provide useful insights for understanding water governance in China. However, the discussion does not fully exploit the results of the case study analysis. The discussion should be more supportive and solid in its content of the results. Also, the narrative analysis of the discussion is very interesting, but if one of the originalities of this study is the identification of change points, more discussion of the triggers that caused these change points is needed. Please refer to my review comments below and revise them to enhance the value of this paper.

\AR{} Thanks!

\subsection{Comment \#1}
% 图1:很难理解B中每个数字的含义。也许是这个数字的大小有问题。
\RC{} Figure 1: It is difficult to understand what each number in B means. Perhaps it may be a problem with the size of the figure.

\AR{} Thanks for the comment. We now improved the figure size in the revised manuscript.

\subsection{Comment \#2}
% 导言:与流域和水资源相关的指标有很多,其中一些指标具有与治理和政策相关的参数。为了突出你所制定的IWGI的新颖性,也许你可以加上对其他相关指标的审查。
\RC{} Introduction: There are many indicators related to watersheds and water resources, some of which have parameters related to governance and policy. To highlight the novelty of the IWGI that you have developed, perhaps you could add a review of other relevant indicators.

\AR{} Thanks for valuable suggestion. We now added a whole paragraph to the introduction:  % TODO 检查引用

\begin{quote}
    Responding the questions for better water governance practices, previous studies can divide into two categories of comprehensive indicators by focusing on water systems or governance systems.
    From the governance side, an integrated index can explicitly denote what practices are effecting water governance.
    However, a comprehensive and detailed list of the key components for assessment of water governance is rare, hence quantitative studies usually propose indexes by using proxy dataset of human activities as a simpler way~\cite{varis2019}.
    A qualitative approach to the assessment of water governance can list more elements regarding to water governance, where OECD's index and assessment dataset are the most widely used. % todo cite OECD
    From the water side, intuitive indexes can provide an embodiment of governance outcomes.
    Water stress is the most widely concerned aspect, which has been far developed in comprehensive index when incorporating human's regulation step by step.
    As instances, traditional water stress index only demands and supply, water scarcity index by involving accessibility, and SFV-index further integrated flexibility.
    These easy-to-use indexes of water stress, however, seem to have difficulty in providing a comprehensive characterization of the above water governance aspects without concerning its social usage (i.e., water use purpose and water allocation).
    Standing from water side for comprehensively quantify governance outcomes, here we aim to develop an integrated water governance index to identify its changes by involving apportions of water uses between regions and sectors.
\end{quote}

\subsection{Comment \#3}
% 方程:方程(7)中的CEM是什么?还有,水的比例指的是什么?支持信息中公式(2)中的一些术语没有解释。例如:R或WU。
\RC{} Equations: what is the CEM in equation (7)? Also, what does the water proportion refer to? Some of the terms in equation (2) in Supporting Information are not explained.\ e.g. R or WU.\

\AR{} Thanks for the comment. ``CEM'' should be ``AEM'' -the indicator we applied for IA.\ This has been revised in the new manuscript:

\begin{quote}
    \begin{equation}
		I_A = AEM = \sum_{i=1}^N - \log(p_{i}) * p_{i}
	\end{equation}
\end{quote}

\AR*{} Also, sorry about the misunderstanding term. We modified the sentence with ``water proportion'' to:

\begin{quote}
    where $p_{i}$ is the proportion of regional water use in $i$ to water use of the whole basin (here, $N=4$ considering divided regions in the YRB, see \textit{Supporting Information S1}).
\end{quote}

% 很抱歉关于SFV计算的部分也造成了困惑。实际上我们提到了这些符号的意义,只不过是在所有公式之后。这的确会造成误解,因此在返修后的文章中我们将这段叙述放到了公式之前:
\AR*{} Sorry that the part about SFV calculation also caused confusion. We actually mentioned the meaning of these symbols, but only after all the equations since they shared the same symbols. This indeed can be misleading, so in the revised article we put this statement before the formula:

\begin{quote}
    In all the equations following, $R_{i, avg}$ is the average runoff in region $i$, $RC_i$ is the total storage capacities of reservoirs in the region $i$, $R_{i, std}$ is the standard deviation of runoff in the region $i$.
\end{quote}

\subsection{Comment \#4}
% 第2.2节:您提到您将Pettitt(1979)方法应用于“水文时间序列”,但结果似乎适用于IWGI时间序列数据。请说明您将该方法应用于哪些数据。
\RC{} Section 2.2: You mention that you apply the Pettitt (1979) approach to ``hydrological time series'', but the results appear to be for IWGI time series data. Please clarify to which data you applied the approach.

\AR{} Thanks for pointing it out! Pettitt approach was applied to IWGI time series data. We now clarify that in the method section:

\begin{quote}
    We applied the Pettitt (1979) approach to detect change-points of IWGI within continuous data, since this method has no assumptions about the distribution of the data~\cite{pettitt1979}.
\end{quote}

\subsection{Comment \#5}
% 图3:D中哪些类型的官方文件被统计以及如何统计?
\RC{} Figure 3: What types of official documents in D are counted and how?

% 抱歉之前的手稿中没有更加明确地进行说明,这份文件统计数据源来自黄河的流域管理部门-黄河水利委员会(YRCC),该部门会对黄河相关的“大事”进行记录并编纂发布。本研究仅根据这些“大事”的发生年份进行逐年统计记录数量。
\AR{} Sorry for not being more explicit in the previous manuscript. The document comes from the Yellow River Conservancy Commission (YRCC), the agency that manages the YRB, which records and compiles ``major events'' related to the river. This study only recorded the number of these ``events'' year by year according to the time they occurred.

\AR*{} To make that more explicit, we modified the corresponding description in the Supplementary Materials:

\begin{quote}
    In addition, for capturing more specific water governance practices in the YRB had been recorded in ``Yellow River Events'' by the YRCC;\ we collected these documents from \url{http://www.yrcc.gov.cn/hhyl/hhjs/}.
    The original documents of those ``Events'' come from the Yellow River Conservancy Commission (YRCC), the agency that manages the YRB, which records and compiles ``major events'' related to the river. This study only recorded the number of these ``events'' year by year according to the time they occurred.
\end{quote}

\subsection{Comment \#6}
% 讨论:每个政权的故事都令人印象深刻。然而,结果与图3的关系并不清楚。请通过更直接地引用结果和图3来加强你的论点。
\RC{} Discussion: The stories of each regime are very impressive. However, the relation to the results and Figure 3 is not clear. Please strengthen your argument by citing the results and Figure 3 more directly.

\AR{} Thanks for the suggestion. We now append refers in discussion when figure 3 is supporting:

\begin{quote}
    During the massive supply regime with lower water stress ($1965 \sim 1978$ in the YRB), water governance thus tended to boost water supply for services (mainly provisioning purposes then -livestock and crops) by constructing reservoirs and channels (Figure~\ref{fig:Causes}~B).
\end{quote}

\begin{quote}
    The rapid expansion of irrigated farmland and water diversion facilities in the same decade brought the overburdened YRB close to a critical point (Figure~\ref{fig:Causes}), where increasing supply to meet demand was impractical~\cite{loch2020}.
\end{quote}

\begin{quote}
    The start of the governance transforming regime (P2: $1979 \sim 2001$) coincided with rising competition for water use after the ``reform and opening-up'' (Figure~\ref{fig:Causes}~C).
\end{quote}

\begin{quote}
    As a pioneer in shifting governing institutions, the YRB triggered institutional changes during this regime. These include, for example, slowing the growth of irrigated acreage and leading water-saving infrastructure (Figure~\ref{fig:Causes});
\end{quote}

\subsection{Comment \#7}
% 讨论:从您对变更点的分析中可以明显看出,每个制度中都存在变更点。但为什么会出现这些变化“点”呢?是什么导致或触发了它们?这是管理机构的变化还是象征性的政策变化?该制度期间的特征已被很好地描述,但请更多地考虑为什么变化点发生在特定的年份/时期。
\RC{} Discussion: The existence of change points in each of the regimes is evident from your analysis of the change points. But why did these change ``points'' occur? What caused or triggered them? Was it a change in governing institutions or a symbolic policy change? The characteristics during that regime period are well described, but please give more thought to why the change points occurred in that particular year/period.

% 感谢如此有趣的评论,虽然制度在特定的点发生变化,但我们总体认为制度转变还是一段时间内持续、积累的变化在特定的点产生了统计特征,很难将其总结成某种因素造成的变化。这也是为什么我们要把各种因素考虑在内,构造这样一个综合指数的原因,不是吗?
\AR{} Thanks to such interesting comments, although the result changes at certain points statistically, we generally believe that the change of the system is still a result of the accumulation of water governance practices in period of time. Therefore, it is difficult to summarize it into a change caused by one factor. That's why we have to take all these factors into account and construct a composite index like this, right?

% 我们认为这类似于复杂系统的涌现,许多数不胜数的治理措施在不断地提出与执行(正如图3C所示),每一个措施都会针对一个具体的问题,但流域整体将向何处发展取决于这些措施的合力。
\AR*{} We think of this as analogous to the ``emergence'' of complex systems. Numerous practices being proposed and implemented (e.g., basin-scaled practices of the Yellow River as shown in Figure~3C). Each practice addresses a specific issue, but how the water governance feature changing overall depends on a combination of their outcomes.

% 当然,并不是说完全无法在特定年份为Regime shift给出解释。在中国的治理模式里,``领会精神''是一个有趣的东西。通常,中央高层会释放一些信号,各级政府(包括部委和地方行政部门)如果按照这种``精神''执行了某种治理措施并取得不错的成果,便是一个非常好的政绩。这种治理模式便是我们在结果、图3、讨论中都提及``social atmosphere''(返修前的手稿我们称之为 social transformation)的原因,这种治理风气会引导治理措施。
\AR*{} It is not to say that there is absolutely no explanation for Regime shift in a given year. In China's governance model, ``central spirit'' is interesting. Usually, the top level of the central government sends some signals. If governments at all levels (including ministries and local administrative departments) implement certain governance measures in accordance with this ``spirit'' and achieve good results, it will be a good governance. This feature is the reason why we mentioned the ``social atmosphere'' (the manuscript before the revision was called social transformation) in the results, Figure 3, and discussion -atmosphere can guide the governance practices.

% 例如在上世纪1960s到1970s,中国在毛泽东领导下有着强烈的``征服自然''风气,这是非常出名的,这种风气能导致建立更多的水库大坝、增大供给。而改革开放带来的阶段二的需求增加、水竞争、效率提升。2000年后中国强调环境治理,等等。可以说我们识别的两次转变都和``social atmosphere''的变化不无关系。但我们很小心的避免了这种归因,因为这种治理转变,绝非单独自上而下的一句口号就能实现的,也有很多其它类似的``中央精神''未能得到有效贯彻执行。
\AR*{} For instance, the ``reform and opening up'' led to increased water demand, water competition and efficiency in transformation regime. After 2000, China emphasized environmental issues \dots It can be said that the two changes we identified are partially related to the change of ``social atmosphere''. But we are very careful to avoid this simple attribution, because this change in governance can not be achieved by a top-down slogan alone, and there are many other similar ``central spirit'' has not been effectively implemented.

\AR*{} Taken together, we believe regime shifts are ``emergence'' of the human-water system, but the significant and solid changing points of the YRB can be related to social atmosphere.

\subsection{Comment \#8}
% 图4:A中的蓝色圆形箭头和B中的红色圆形箭头是什么意思?它们似乎包含了您非常重要的言论,但它们并不清楚。此外,您需要更多地解释A和B中的每个箭头。
\RC{} Figure 4: What do the blue circular arrows in A and the red circular arrows in B mean? They seem to contain your very important remarks, but they are unclear. Also, you need to explain more about each of the arrows in A and B.

% 很抱歉之前的手稿造成了此困惑,我们现在在返修后的手稿中增加了以下内容,以便更清晰地说明
\AR{} I'm sorry for the confusion caused by the previous manuscript. We have now added the following content to the caption of that figure in revised manuscript:

\begin{quote}
    Transition schema in hydrosocial cycle and water governance regimes. The natural water cycle dominates blue pathways, while socio-economic feedback dominates red.
    The large circular arrows indicate the social and hydrological processes that dominate in different stages.
    Provisioning water includes water used by human, livestock and crops while non-provisioning water includes used by industry and service.
    The processes expressed in the graph mainly include: Water supports people in provisioning ways or influence/influenced socioeconomic systems as non-provisioning ways; People manage/govern water system based on their well-beings.
    The gray thick arrow represents weaker process, while the red arrow represents more significant one.
\end{quote}

\AR*{} In addition, other two sentences in discussion can make it clearer:

\begin{quote}
    To support water use in early stage (Figure~\ref{fig:summary}~A), strategies tend to manage natural water processes in order to maintain the provisioning (larger) and non-provisioning water (less).
    At the later stage (Figure~\ref{fig:summary}~B), emphasis is governing across the whole basin, water governance practices are adaptively designed to meet the increasing needs of the socio-economic system and carried out.
    With the above gradually shifting, the emergence of different regimes drives water governance challenges at a basin-scale: these were primarily economic and environmental before the transformation, but social and policy-related towards the end (Figure~\ref{fig:summary})~\cite{singh2019,porcher2019}.
\end{quote}

\subsection{Comment \#9}
% 结论:对定量分析的结果进行了描述,但是Discussion中的叙事分析似乎没有得到充分的体现。建议对此添加几行。
\RC{} Conclusion: The results of the quantitative analysis are described, but the narrative analysis in the Discussion does not seem to be fully reflected. It is recommended to add a few lines about it.

\AR{} Sure! Thanks for the valuable suggestion. We now added narrative analysis in the discussion to our conclusion:

\begin{quote}
    During the massive supply regime (P1: $1965 \sim 1978$), water governance tended to boost water supply by constructing reservoirs and channels in the YRB.\
    Then, the start of the governance transforming regime (P2: $1979 \sim 2001$) coincided with rising competition for water use and led to institutional changes like water-saving improvements, water quota policies, and cross-boundary water transfer plans.
    Last, adaptation-oriented regime (P3: $2002 \sim 2013$) with stable high water stress resulted in trade-offs and joint regulating between water-dependent regions and sectors.
\end{quote}
