
\subsection*{Overall comments}
% 作者提出了IWGI,并将其应用于黄河流域的水治理历史,分析了流域的制度变迁。该方法具有很强的合理性和适应性,案例分析的结果为理解中国的水治理提供了有益的见解。然而,讨论并没有充分利用案例研究分析的结果。讨论结果的内容应该更具有支持性和可靠性。此外,讨论的叙事分析非常有趣,但如果这项研究的原创性之一是识别变化点,那么就需要更多地讨论导致这些变化点的触发因素。请参考我下面的审稿意见并进行修改,以提高本文的价值。
\RC{} The authors proposed the IWGI, applied the index to the history of water governance in the Yellow River basin, and analyzed the regime change in the basin. The methodology is very reasonable and adaptable, and the results of the case analysis provide useful insights for understanding water governance in China. However, the discussion does not fully exploit the results of the case study analysis. The discussion should be more supportive and solid in its content of the results. Also, the narrative analysis of the discussion is very interesting, but if one of the originalities of this study is the identification of change points, more discussion of the triggers that caused these change points is needed. Please refer to my review comments below and revise them to enhance the value of this paper.

\AR{} Thanks!

\subsection{Comment \#1}
% 图1:很难理解B中每个数字的含义。也许是这个数字的大小有问题。
\RC{} Figure 1: It is difficult to understand what each number in B means. Perhaps it may be a problem with the size of the figure.

\subsection{Comment \#2}
% 导言:与流域和水资源相关的指标有很多,其中一些指标具有与治理和政策相关的参数。为了突出你所制定的IWGI的新颖性,也许你可以加上对其他相关指标的审查。
\RC{} Introduction: There are many indicators related to watersheds and water resources, some of which have parameters related to governance and policy. To highlight the novelty of the IWGI that you have developed, perhaps you could add a review of other relevant indicators.

\subsection{Comment \#3}
% 方程:方程(7)中的CEM是什么?还有,水的比例指的是什么?支持信息中公式(2)中的一些术语没有解释。例如:R或WU。
\RC{} Equations: what is the CEM in equation (7)? Also, what does the water proportion refer to? Some of the terms in equation (2) in Supporting Information are not explained. e.g. R or WU. 

\subsection{Comment \#4}
% 第2.2节:您提到您将Pettitt(1979)方法应用于“水文时间序列”,但结果似乎适用于IWGI时间序列数据。请说明您将该方法应用于哪些数据。
\RC{} Section 2.2: You mention that you apply the Pettitt (1979) approach to "hydrological time series", but the results appear to be for IWGI time series data. Please clarify to which data you applied the approach.

\subsection{Comment \#5}
% 图3:D中哪些类型的官方文件被统计以及如何统计?
\RC{} Figure 3: What types of official documents in D are counted and how?

\subsection{Comment \#6}
% 讨论:每个政权的故事都令人印象深刻。然而,结果与图3的关系并不清楚。请通过更直接地引用结果和图3来加强你的论点。
\RC{} Discussion: The stories of each regime are very impressive. However, the relation to the results and Figure 3 is not clear. Please strengthen your argument by citing the results and Figure 3 more directly.

\subsection{Comment \#7}
% 讨论:从您对变更点的分析中可以明显看出,每个制度中都存在变更点。但为什么会出现这些变化“点”呢?是什么导致或触发了它们?这是管理机构的变化还是象征性的政策变化?该制度期间的特征已被很好地描述,但请更多地考虑为什么变化点发生在特定的年份/时期。
\RC{} Discussion: The existence of change points in each of the regimes is evident from your analysis of the change points. But why did these change "points" occur? What caused or triggered them? Was it a change in governing institutions or a symbolic policy change? The characteristics during that regime period are well described, but please give more thought to why the change points occurred in that particular year/period.

\subsection{Comment \#8}
% 图4:A中的蓝色圆形箭头和B中的红色圆形箭头是什么意思?它们似乎包含了您非常重要的言论,但它们并不清楚。此外,您需要更多地解释A和B中的每个箭头。
\RC{} Figure 4: What do the blue circular arrows in A and the red circular arrows in B mean? They seem to contain your very important remarks, but they are unclear. Also, you need to explain more about each of the arrows in A and B.

\subsection{Comment \#9}
% 结论:对定量分析的结果进行了描述,但是Discussion中的叙事分析似乎没有得到充分的体现。建议对此添加几行。
\RC{} Conclusion: The results of the quantitative analysis are described, but the narrative analysis in the Discussion does not seem to be fully reflected. It is recommended to add a few lines about it.

\AR{} answer
