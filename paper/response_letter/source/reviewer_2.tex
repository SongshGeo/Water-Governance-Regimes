\section{Reviewer \#2}\label{reviewer_2}

\subsection*{Reviewer \#2 Evaluations:}

\begin{itemize}
    \item Recommendation (Required): Return to author for minor revisions
    \item Significant: The paper has some unclear or incomplete reasoning but will likely be a significant contribution with revision and clarification.
    \item Supported: Yes
    \item Referencing: Yes
    \item Quality: Yes, it is well-written, logically organized, and the figures and tables are appropriate.
    \item Data: Yes
    \item Accurate Key Points: Yes
\end{itemize}

\subsection*{General comments}

\RC{} Thank you for sharing your revised manuscript. The revised manuscript was very carefully constructed. This revision solved most of my questions, but there were some incomplete and vague parts in the response to ``2.7. Comment \#7''.

\RC*{} I agree that the change of the system is still a result of the accumulation of water governance practices in period. However, it was clear from your results and discussion that these system changes are dictated by specific factors (historical events): e.g., the Chinese slogan ``human will conquer nature'', rapid expansion of irrigated farmland and water diversion facilities so on, in the P1. I suggest that you organize them in a table that shows those factors (historical events) that are causing the ``emergence'' of each complex system. (Figure 4 is too conceptual for the historical phenomenon you were describing.) I believe that this will make the characteristics of each regime clearer.

\RC{} The ``social atmosphere'' is an interesting term, but very vague. First of all, please define what it is (I have never heard of such a term in previous studies in my field). At the very least, you should discuss what ``social atmosphere'' is in the Discussion. Furthermore, I really wonder how your research and the proposed index can be practically contribute to understanding ``the successes and failures of water governanceis'' created by the ``social atmosphere''. Perhaps this ``social atmosphere'' is the most troubling and important factor to understand in water governance regime.

\AR{} Thank you for raising an intriguing point of discussion on the concept of ``social atmosphere''. We have taken your feedback into account and provided a definition in the revised manuscript to elucidate this term:

\begin{quote}[page=footnotes from Figure~3 and Table~1]
    Social atmosphere refers to the sociocultural context in which people live or in which something happens, including the culture that the individual was educated or lives.
\end{quote}

\AR*{} We maintain a cautious stance regarding the assertion that changes in the ``social atmosphere'' directly lead to shifts in stable states, as our findings do not establish a causal relationship between the two. Nevertheless, as depicted in the newly added Table~1, prompted by your suggestion, a temporal co-occurrence between the two is observable. We discussed it more in the revised manuscript:

\begin{quote}[page=10, sline=265, eline=269]
    These regime shifts, as comprehensive outcomes of complex human-water interactions, are coincidentally well presented by several well-known and dominated social atmospheres (Table~1).
    The social atmospheres refers to the sociocultural context in which people live or in which something happens, underlying the direction of practices of water governance during a regime, despite lack of strict causality evidence.
\end{quote}

\AR*{} The evolution of the ``social atmosphere'' could potentially be a consequence of changes in human-water relations or a catalyst for such transformations. We appreciate your perspective that this shift might represent a breakthrough in addressing complex challenges in future studies. We have enriched our discussion with additional insights to reflect on this complex interplay. However, for the scope of this current study, the evidence at our disposal does not permit an extensive exploration centered on this aspect. It's interesting to delve deeper into this intricate relationship in our future work, backed by comprehensive data and analyses.

\subsection*{Minor comment}

\RC{} I couldn't access the link http://www.yrcc.gov.cn/hhyl/hhjs/ in the Supporting Informations. Please confirm it.

\AR{} Sorry to hear that. However, we double-checked that this link is still accessible. It's a page from the official website of Yellow River Conservancy Commission (YRCC). We also uploaded the dataset ``big\_events.csv'' of the web page is made by using the web crawler in the open-sourced Zenodo repo.
