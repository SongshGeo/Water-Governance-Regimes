% chktex-file 46
\section*{Associate Editor}\label{editor}

\subsection{General comments}
\RC{} Dear Authors, thanks for the submission of the manuscript. We have asked two experts in the relevant fields to review the manuscript. Both reviewers overall find the merit in the manuscript however they also noted several major issues in the current manuscript. With my own reading, I fully agree with the reviewers and would like to highlight a few major issues:

\AR{} Thank you for your constructive comments and for providing us with the opportunity to address the concerns raised by both reviewers. We appreciate the valuable feedback, which has helped us identify areas for improvement in our manuscript. We have carefully considered each point and have made substantial revisions to address these issues. Please find our detailed responses to the highlighted concerns below.

\subsection{Issue \#1}
\RC{} overall the results and discussion are relatively shallow and I would encourage the authors to do deep dive into the results, e.g.\ change points, as also pointed out by both reviewers,

\AR{} We acknowledge the need for a more in-depth analysis of our previous results. We have re-evaluated our data and expanded the results and discussion sections to include a deeper dive into the findings, specifically regarding the change points mentioned by both reviewers. We have also provided more context, explanation, and robustness analysis in the Supporting Informations (Section S2 to Section S4). There is a detailed explanation for this issue in the Response~\ref{sec:1-2} to the reviewer \#1 and the Response~\ref{sec:2-7} to the reviewer \#2.

\subsection{Issue \#2}
\RC{} I fully agree with Reviewer \#2 that more thorough review of existing and relevant indicators with IWGI,

\AR{} We agree with Reviewer \#2's suggestion to include a more thorough review of existing and relevant indicators with IWGI.\ We have added a new paragraph in the introduction discussing the existed indexes, comparing and contrasting their strengths and weaknesses, and explaining innovation of IWGI.\ There is a detailed explanation for this issue in the Response~\ref{sec:2-2} to the reviewer \#2.

\subsection{Issue \#3}
\RC{} I would encourage the authors to look into more recent data (beyond 2013) as pointed out by Reviewer 1. In addition, I find the data in Supporting Informations is not accessible, which should be amended. The data access and reproduction of the methodology/results are very important for WRR and our scientific community, so we take the issue related with data accessibility very seriously.

\RC*{} Overall, based on the reviewers' recommendations, I would like to ask the authors to submit a suitably revised manuscript in due course.

\AR{} We appreciate the importance of using up-to-date data in our research. We have now included an analysis with more recent data in our supporting information. You can find a detailed explanation for this issue in the Response~\ref{sec:1-3} to the reviewer \#1.

\AR*{} We apologize for the inconvenience caused by the inaccessibility of the dataset. We have rectified this issue by refined our data description. Now, all the used datasets and their links are detailed described in the method section:

\begin{quote}
	For calculating IWGI, three datasets were used: reservoirs, measured runoff, and water uses.
	The reservoir dataset was collected by~\citeA{wang2019c}, which introduced includes the significant new reservoirs built in the YRB since 1949 (Accessible with reasonable request).
	Among all the reservoirs, YRCC labelled the ``major reservoirs'' which were constructed mainly for regulating and managing (see \url{http://www.yrcc.gov.cn/hhyl/sngc/}).
	In addition, annual measured runoff data was collected from the Yellow River Sediment Bulletin (\url{http://www.yrcc.gov.cn/nishagonggao/}) and four controlling stations are measuring different reaches of the Yellow River (see Supporting Informations Section~S1).
	The water resources use dataset was from National Long-term Water Use Dataset of China (NLWUD) published by \citeA{zhou2020}, which includes water uses, water-consuming economic variables, and water use intensities by sectors the prefectures level.
	We determined the prefectures belong to the YRB by filtering the NLWUD dataset with a threshold of $95\%$ intersected area.

	For analyzing its causes of changing water, irrigated area, gross added values of industry and services, and water use intensities data were also from NLWUD dataset~\cite{zhou2020}.
	Besides, two water governance policies datasets are used: laws data and ``big events'' documents dataset.
	Data of laws were collected from~\citeA{yellowriverwaterconservancycommission2010}, which reviewed all important laws at the basin scale related to the Yellow River from the last century.
	The original documents of ``big events'' related to the Yellow River come from the YRCC, the agency at the basin scale, which recorded and compiled these events (\url{http://www.yrcc.gov.cn/hhyl/hhjs/}).

	Finally, we calculated the IWGI from 2001 to 2017 (the latest) in the Supporting Information Section S4 for robustness test with another water use dataset from Yellow River Water Resources Bulletin (\url{http://www.yrcc.gov.cn/zwzc/gzgb/gb/szygb/}).
\end{quote}

\AR*{} We hope that these revisions address the concerns raised by both reviewers. We are confident that these changes have significantly improved the quality of our manuscript and made it more relevant and valuable to the scientific community. We look forward to your further feedback and the possibility of our manuscript being accepted for publication.
