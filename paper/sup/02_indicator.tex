By taking water flexibility and variability into account, the scarcity-flexibility-variability (SFV) index focus more on dynamic responses to water resources in a developing perspective, which is a valid metric of temporal changes in water stresses \cite{qin2019}. To apply this method, we need to combine three metrics following:

First, for scarcity, $A_{i, j}$ is the total water consumption as a proportion of regional multi-year average runoff volume in year $j$ and region $i$ (in this study, four regions in the YRB, \textit{Appendix}~\ref{secA1}):
\begin{equation}
    A_{i, j} = \frac{WU_{i,j}}{R_{i, avg}}
\end{equation}

Second, for flexibility, $B_{i, j}$ is the inflexible water use $WU_{inflexible}$ (i.e. for thermal power plants or humans and livestock) as a proportion of average multi-year runoff, in year $i$ and region $j$:
\begin{equation}
    B_{i, j} = \frac{WU_{i, j, inflexible}}{R_{i, avg}}
\end{equation}

Finally for variability, the capacity of the reservoir and the positive effects of storage on natural runoff fluctuations are also considered.
\begin{gather}
C_i = C1_i * (1 - C2_i) \\
C1_{i, j} = \frac{R_{i, std}}{R_{i, avg}} \\
C2_{i} = \frac{RC_{i}}{R_{i, avg}}, \ if RC < R_{i, avg} \\
C2_{i} = 1, \ if RC >= R_{i, avg}
\end{gather}

In all the equations above, $R_{i, avg}$ is the average runoff in region $i$, $RC_i$ is the total storage capacities of reservoirs in the region $i$, $R_{i, std}$ is the standard deviation of runoff in the region $i$.

Finally, assuming three metrics (scarcity, flexibility and variability) have the same weights, we can calculate the $SFV$ index after normalizing them:
\begin{gather}
    V = \frac{A_{normalize} + B_{normalize} + C_{normalize}}{3}\\
    a = \frac{1}{V_{max} - V_{min}};\\
    b = \frac{1}{V_{min} - V_{max}} * V_{min}\\
    SFV = a * V + b
\end{gather}
