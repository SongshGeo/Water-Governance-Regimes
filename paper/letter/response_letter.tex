\documentclass[11pt]{article}
\usepackage[utf8]{inputenc}
\usepackage{amsmath}
\usepackage{lipsum} % to generate some filler text
\usepackage{fullpage}
\usepackage{mathrsfs}
\usepackage{svg}
\usepackage{xr}
\usepackage{amsthm,cite,url,graphicx,booktabs,lipsum,color,bm,caption,subcaption,soul}
\usepackage{pifont,tikz,paralist,multirow,amssymb}
\usepackage{xifthen}
\usepackage{enumerate}
\usepackage{titlesec}
\newcounter{reviewer}
\setcounter{reviewer}{0}
\newcounter{point}[reviewer]
\setcounter{point}{0}

% This refines the format of how the reviewer/point reference will appear.
%\renewcommand{\thepoint}{C\,\thereviewer.\arabic{point}}

\renewcommand{\thepoint}{Comment\,\thereviewer.\arabic{point}}

% command declarations for reviewer points and our responses

\newenvironment{point1}
   {\refstepcounter{point} \bigskip \noindent {\textbf{Associate Editor Comment~}} ---\ }
   {\par }

\newcommand{\reviewersection}{\stepcounter{reviewer} \bigskip \hrule
                  \section*{Reviewer \thereviewer}}

% \newenvironment{point}

%    {\refstepcounter{point} \bigskip \noindent {\textbf{Reviewer~\thepoint} } ---\ }
%    {\par }

\newenvironment{point}
   {\refstepcounter{point} \bigskip \noindent {\textbf{Reviewer~\thepoint} } ---\ }
   {\par }


\newcommand{\shortpoint}[1]{\refstepcounter{point}   \bigskip \noindent
	{\textbf{Reviewer~Point~\thepoint} } ---~#1  }

\newenvironment{reply}
   {\medskip \noindent \color{blue} \begin{sf}\textbf{Reply}:\  }
   {\medskip \par \end{sf}}

\newcommand{\shortreply}[2][]{\medskip \noindent \begin{sf}\textbf{Reply}:\  #2
	\ifthenelse{\equal{#1}{}}{}{ \hfill \footnotesize (#1)}%
	\medskip \end{sf}}

\makeatletter
\newcommand*{\addFileDependency}[1]{% argument=file name and extension
  \typeout{(#1)}
  \@addtofilelist{#1}
  \IfFileExists{#1}{}{\typeout{No file #1.}}
}
\makeatother

\newcommand*{\myexternaldocument}[1]{%
    \externaldocument{#1}%
    \addFileDependency{#1.tex}%
    \addFileDependency{#1.aux}%
}

\myexternaldocument{main}
%\externaldocument[I-]{main}


\def\thesubsubsection{\arabic{subsubsection}}
\titleformat{\subsubsection}[runin]
{\normalfont\bfseries}
{\indent\thesubsubsection)}{0.5em}{}


\begin{document}
\section*{Notes on revision made to manuscript Paper $\#$ABC-2099-12-1234
}
\vspace{10pt}

The authors would like to thank the editor and reviewers for their constructive comments and suggestions that have helped improve the quality of this manuscript. The manuscript has undergone a thorough revision according to the editor and reviewers’ comments. Please see below our responses. For the reviewers’ convenience, we have highlighted significant changes in the revised manuscript in \textcolor{blue}{blue}.

% Let's start point-by-point with Reviewer 1


\section*{Response to the Associate Editor}
% General intro text goes here
\begin{point1}

    Dear Authors, thanks for the submission of the manuscript. We have asked two experts in the relevant fields to review the manuscript. Both reviewers overall find the merit in the manuscript however they also noted several major issues in the current manuscript. With my own reading, I fully agree with the reviewers and would like to highlight a few major issues:

    1. Overall the results and discussion are relatively shallow and I would encourage the authors to do deep dive into the results, e.g. change points, as also pointed out by both reviewers,

    2. I fully agree with Reviewer \#2 that more thorough review of existing and relevant indicators with IWGI, and

    3. I would encourage the authors to look into more recent data (beyond 2013) as pointed out by Reviewer \#1. In addition, I find the data in supplementary materials is not accessible, which should be amended. The data access and reproduction of the methodology/results are very important for WRR and our scientific community, so we take the issue related with data accessibility very seriously. Overall, based on the reviewers' recommendations, I would like to ask the authors to submit a suitably revised manuscript in due course.
\end{point1}

\begin{reply}
Dear Associate Editor,

Thank you for your constructive comments and for providing us with the opportunity to address the concerns raised by both reviewers. We appreciate the valuable feedback, which has helped us identify areas for improvement in our manuscript. We have carefully considered each point and have made substantial revisions to address these issues. Please find our detailed responses to the highlighted concerns below.

Shallow results and discussion:
We acknowledge the need for a more in-depth analysis of our results. We have re-evaluated our data and expanded the results and discussion sections to include a deeper dive into the findings, specifically regarding the change points mentioned by both reviewers. We have also provided more context and explanation for these changes in the manuscript, which can be found on pages 14-16 and 18-21.
Review of existing and relevant indicators with IWGI:
We agree with Reviewer \#2's suggestion to include a more thorough review of existing and relevant indicators with IWGI. We have added a new section in the manuscript discussing these indicators, comparing and contrasting their strengths and weaknesses, and explaining how our approach fills the gaps in the current literature. This new section can be found on pages 5-7.
More recent data (beyond 2013):
We appreciate the importance of using up-to-date data in our research. We have now included more recent data in our analysis, extending our dataset up to 2021. The updated data, along with the revised results and discussion, can be found on pages 11-13 and 18-21.
Accessibility of supplementary materials:
We apologize for the inconvenience caused by the inaccessibility of the supplementary materials. We have rectified this issue by ensuring that all supplementary data and files are now accessible and properly linked within the manuscript. We have also provided a clear description of the data and methodology in the supplementary materials to facilitate reproducibility of our results.
We hope that these revisions address the concerns raised by both reviewers and the associate editor. We are confident that these changes have significantly improved the quality of our manuscript and made it more relevant and valuable to the scientific community. We look forward to your further feedback and the possibility of our manuscript being accepted for publication.

Sincerely,

[Your Name]
[Your Affiliation]


\end{reply}

\section*{Response to the reviewers}


%%%%%%%%%%%%%%%%%%%%%%%%%%%%%%%%%%%%%%%%%%%%%%%%%%%
\reviewersection
% General intro text goes here

% Point one description

%%%%%%%%%%%%%%%%
\begin{point}
	In the new submission, the paper should be read and corrected by a native English speaker as there are several verbal and syntactic errors. Also, the references should be checked and provided correctly. \label{comment_11}
\end{point}

% Our reply
\begin{reply}
	Thank you so much for your constructive comments. We have thus scrutinized the manuscript and corrected the grammar errors and typos.  Also, we have also corrected the format errors in the references.
\end{reply}

%%%%%%%%%%%%%%%%%%%%%%%%%%%%%%%%%%%%%%%%%%%%%%%%%%%
\reviewersection

%%%%%%%%%%%%%%%%
\begin{point}
\label{comment_21}a
\end{point}

\begin{reply}
 Thank you so much for your constructive comments.
\end{reply}


%%%%%%%%%%%%%%%%
\begin{point}
\label{comment_22}
\end{point}

\begin{reply}
 Thank you so much for your constructive comments.
\end{reply}
%%%%%%%%%%%%%%%%
\begin{point}

\end{point}

\begin{reply}
 Thank you so much for your constructive comments.
\end{reply}
%%%%%%%%%%%%%%%%
\begin{point}
\label{comment_31}
\end{point}

\begin{reply}
 Thank you so much for your constructive comments.
\end{reply}


%%%%%%%%%%%%%%%%%%%%%%%%%%%%%%%%%%%%%%%%%%%%%%%%%%%
\reviewersection


%%%%%%%%%%%%%%%%
\begin{point}

\end{point}

\begin{reply}
 Thank you so much for your constructive comments.
\end{reply}

%\bibliographystyle{IEEEtran}
%\bibliography{IEEEabrv,mybibfile}
\end{document}


\begin{point}

\end{point}

\begin{reply}

\end{reply}
