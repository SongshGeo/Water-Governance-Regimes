\documentclass[11pt,a4paper,roman]{moderncv}
\usepackage[english]{babel}
\usepackage{ragged2e}
\usepackage{float}
\usepackage{graphicx}
\usepackage[utf8]{inputenc}


\moderncvstyle{classic}
\moderncvcolor{green} % Bullet point color

% Page margins
\usepackage[scale=0.8]{geometry} % Page margins

% Your Information, please revise
\name{Shuang}{Song}
\address{Haidian District, Beijing, China}{No.19, Xinjiekouwai St.}
\phone[mobile]{+86 185 0068 5922}
\email{songshgeo@mail.bnu.edu.cn}


\begin{document}

% Insert Olin Logo
\begin{minipage}[t]{\textwidth}
\includegraphics[width=0.40\textwidth]{bnu}
\end{minipage}


\recipient{Dear Editor in Chief,}{}
\opening{\vspace*{-2em}}
\closing{Sincerely,}{\vspace*{-2em}}
\enclosure[Enclosures]{Main manuscript, Appendix}
\makelettertitle

I am pleased to submit an original research article entitled ``Identifying regime transitions for water governance at a basin scale'' for consideration for publication in \textit{Nature Water} as an \textit{Analysis}.

Water governance determines ``who gets water, when, and how'' in most large river basins; missing governance means missing sustainability. However, the lack of a comprehensive but straightforward approach to identifying the changes in water governance presents a challenge for efforts to underpin it. Therefore, we choose indicators for the corresponding aspects (water stress, water services purpose, and water allocation) and combine them into an integrated water governance index (IWGI) to analyze long-term changes in a large river basin.

% \begin{figure}
% 	\centering
% 	\includegraphics[width=0.9\textwidth]{../../figures/main/framework.png}
% 	\caption{
% 		Identifying the water governance regimes in transitions of a hydrosocial cycle with an integrated water governance index (IWGI). Water stress (S), purposes of water services (P), and water allocation (A) are three aspects to be considered (\textbf{A.}). Along with hydrosocial-cycle transitions, a human-dominated regime influences these aspects of water governance. For example, the construction of reservoirs (1) aims to alleviate water stress; growth of energy and industry (2); water-lead intensive agriculture (3); conveyance system (4) controls water allocation.
% 		Therefore, the methodology is to combine three aspects' corresponding indicators, and then an abrupt change of the IWGI can indicate a regime shift in water governance (\textbf{B.}).
% 	}
% 	\label{fig:framework}
% \end{figure}

Applying the IWGI to the rapidly-changing Yellow River Basin (YRB) in China clarifies shifts in water governance between massive supply, transformation governance, and adaptation-oriented regimes. After interpreting these regimes with underlying causes, we propose a potentially widespread transition schema in a human-dominated hydrosocial cycle.

Our analysis approach links river basin governance with transitions of hydrosocial water cycles and offers valuable sustainability guidelines for big river basins worldwide. Therefore, we believe that this manuscript is appropriate for publication by \textit{Nature Water} because it tightly echoes the aims \& scope of water governance.

This manuscript has not been published and is not under consideration for publication elsewhere, and all the authors have read and approved the manuscript. The manuscript includes the main body of text with $4176$ words and $4$ figures; the three-section appendix has $3$ supporting figures and $2$ supporting tables. We have no conflicts of interest to disclose. If you feel that the manuscript is appropriate for your journal, we suggest the following reviewers:

[list reviewers and contact info, if requested by the journal]

\vspace{0.5cm}

I appreciate your consideration!

\vspace{0.5cm}

Sincerely,

\textbf{Bojie Fu} (On behalf of the author team)

%\begin{minipage}[t]{\textwidth}
%	\includegraphics[height=1cm]{signature.jpeg}
%\end{minipage}

State Key Laboratory of Earth Surface Processes and Resource Ecology,

Faculty of Geographical Science,

Beijing Normal University,

Beijing 100875, China

Email: bfu@rcees.ac.cn

\end{document}
