% chktex-file 10
\documentclass{article}
	\def\papertitle{Identifying regime transitions for water governance at a basin scale}
	\def\authors{Shuang Song, Shuai Wang}
	\def\journal{Water Resources Research}
	\def\doi{\#2022WR033819}
\input{../templates/AR_to_RC.tex}

\section{Associate Editor}

\subsection{General comments}
\RC{} Dear Authors, thanks for the submission of the manuscript. We have asked two experts in the relevant fields to review the manuscript. Both reviewers overall find the merit in the manuscript however they also noted several major issues in the current manuscript. With my own reading, I fully agree with the reviewers and would like to highlight a few major issues:

\AR{} Thank you for your constructive comments and for providing us with the opportunity to address the concerns raised by both reviewers. We appreciate the valuable feedback, which has helped us identify areas for improvement in our manuscript. We have carefully considered each point and have made substantial revisions to address these issues. Please find our detailed responses to the highlighted concerns below.

\subsection{Issue \#1}
\RC{} overall the results and discussion are relatively shallow and I would encourage the authors to do deep dive into the results, e.g.\ change points, as also pointed out by both reviewers,

\AR{} We acknowledge the need for a more in-depth analysis of our results. We have re-evaluated our data and expanded the results and discussion sections to include a deeper dive into the findings, specifically regarding the change points mentioned by both reviewers. We have also provided more context and explanation for these changes in the manuscript, which can be found on pages 14\-16 and 18\-21.

\subsection{Issue \#2}
\RC{} I fully agree with Reviewer \#2 that more thorough review of existing and relevant indicators with IWGI,

\AR{} We agree with Reviewer \#2's suggestion to include a more thorough review of existing and relevant indicators with IWGI. We have added a new section in the manuscript discussing these indicators, comparing and contrasting their strengths and weaknesses, and explaining how our approach fills the gaps in the current literature. This new section can be found on pages 5\-7.

\subsection{Issue \#3}
\RC{} I would encourage the authors to look into more recent data (beyond 2013) as pointed out by Reviewer 1. In addition, I find the data in supplementary materials is not accessible, which should be amended. The data access and reproduction of the methodology/results are very important for WRR and our scientific community, so we take the issue related with data accessibility very seriously.

\RC*{} Overall, based on the reviewers' recommendations, I would like to ask the authors to submit a suitably revised manuscript in due course.

\AR{} We appreciate the importance of using up-to-date data in our research. We have now included more recent data in our analysis, extending our dataset up to 2021. The updated data, along with the revised results and discussion, can be found on pages 11-13 and 18-21.

\AR*{} We apologize for the inconvenience caused by the inaccessibility of the supplementary materials. We have rectified this issue by ensuring that all supplementary data and files are now accessible and properly linked within the manuscript. We have also provided a clear description of the data and methodology in the supplementary materials to facilitate reproducibility of our results.

\AR*{} We hope that these revisions address the concerns raised by both reviewers and the associate editor. We are confident that these changes have significantly improved the quality of our manuscript and made it more relevant and valuable to the scientific community. We look forward to your further feedback and the possibility of our manuscript being accepted for publication.

\newpage
\section{Reviewer \#1}

\subsection{Overall comments}

\RC{} This study proposed an integrated index that incorporates water resources, water use, and water allocation to represent the water governance regime in the Yellow River basin. The authors showed an in-depth understanding of the water governance in this basin and made a great effort to represent the status of water governance straightforwardly with the integrated index. The figures were generally well-designed and presented clearly. But I feel the paper lacks details of the results which makes it difficult to understand the paper. Specific comments are listed below.

% 感谢您认可。
% 很抱歉缺少细节,可能会使您难以完全理解我们的研究。
% 为了回应您的担忧,我们已经修改了我们的稿件,并提供了更详细的结果解释。请在下面找到我们对您的具体评论的逐条回复。
\AR{} Thank you for your valuable feedback and for acknowledging our efforts in understanding the water governance in the Yellow River basin. We appreciate your comments regarding the clarity of the figures and our integrated index. Sorry about the lack of details in the results section may have made it difficult to fully comprehend our study. In response to your concerns, we have revised our manuscript and provided a more detailed explanation of the results. Please find our point-by-point response to your specific comments below.

\subsection{Major concern \#3}

% 气候变化在指数中没有明确的体现,这可能是本世纪黄河流量恢复的关键因素。气候变化将是未来该流域面临的巨大挑战。重大的气候变化可能影响流域的适应能力,因此需要不同的治理策略。
\RC{} Climate change is not explicitly represented in the index, which might play a key role in the streamflow recovery in the Yellow River in this century. Climate change would be a great challenge in the basin in the future. Significant climate change may affect the adaptive capacity of a basin and thus require different governance strategies.

% 气候变化的确会产生很多影响,例如改变降水量和蒸发量。但这些影响都会以影响可用水量的形式体现在IWGI之中。
% 例如气候变化让可用水量发生改变(scarcity),极端气候让我们难以有效的分配水资源(allocation),对气候变化的适应可能导致社会变革,让人们对愈发侧重于工业和经济发展的治理策略进行反思(priority)。
% 我们现在在讨论中涉及了这些话题。
\AR{} Ok, I changed this:

% 我们强调流域治理需要考虑这些因素,但我们可能很难穷尽优秀的流域治理策略需要考虑哪些因素。因此,IWGI至少让我们可以知道在诸多治理因素的共同影响下,流域的未来正在走向何方。
\AR*{}

\subsection{Specific comments \#4}
% 结果是总结性的。我建议提供更多关于三个指数和IWGI结果的信息。例如,三个指标的结果,以及分区域的结果。请解释IS、IP、IA和IWGI这三个指标的范围,以及不同数值的含义。这些细节可以帮助读者理解结果的合理性。
\RC{} The results are summative. I would suggest providing more information on the results of the three indices and the IWGI.\ For example, the results of the three indices, and the results for the sub-regions. And please explain the range of the three indicators, IS, IP, IA, and the IWGI, and the meanings of different values. These details could help readers understand the reasonability of the results.




\end{document}
