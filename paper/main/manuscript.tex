%%%%%%%%%%%%%%%%%%%%%%%%%%%%%%%%%%%%%%%%%%%%%%%%%%%%%%%%%%%%%%%%%%%%%
%%                                                                 %%
%% Please do not use \input{...} to include other tex files.       %%
%% Submit your LaTeX manuscript as one .tex document.              %%
%%                                                                 %%
%% All additional figures and files should be attached             %%
%% separately and not embedded in the \TeX\ document itself.       %%
%%                                                                 %%
%%%%%%%%%%%%%%%%%%%%%%%%%%%%%%%%%%%%%%%%%%%%%%%%%%%%%%%%%%%%%%%%%%%%%

%%\documentclass[referee,sn-basic]{sn-jnl}% referee option is meant for double line spacing

%%=======================================================%%
%% to print line numbers in the margin use lineno option %%
%%=======================================================%%

%%\documentclass[lineno,sn-basic]{sn-jnl}% Basic Springer Nature Reference Style/Chemistry Reference Style

%%======================================================%%
%% to compile with pdflatex/xelatex use pdflatex option %%
%%======================================================%%

%%\documentclass[pdflatex,sn-basic]{sn-jnl}% Basic Springer Nature Reference Style/Chemistry Reference Style

%%\documentclass[sn-basic]{sn-jnl}% Basic Springer Nature Reference Style/Chemistry Reference Style
% \documentclass[sn-mathphys]{sn-jnl}% Math and Physical Sciences Reference Style
%%\documentclass[sn-aps]{sn-jnl}% American Physical Society (APS) Reference Style
%%\documentclass[sn-vancouver]{sn-jnl}% Vancouver Reference Style
%%\documentclass[sn-apa]{sn-jnl}% APA Reference Style
%%\documentclass[sn-chicago]{sn-jnl}% Chicago-based Humanities Reference Style
% \documentclass[sn-standardnature]{sn-jnl}% Standard Nature Portfolio Reference Style
% \documentclass[default]{sn-jnl}% Default
\documentclass[default, sn-standardnature]{sn-jnl}% Default with double column layout

%%%% Standard Packages
%%<additional latex packages if required can be included here>
%%%%

%%%%%=============================================================================%%%%
%%%%  Remarks: This template is provided to aid authors with the preparation
%%%%  of original research articles intended for submission to journals published
%%%%  by Springer Nature. The guidance has been prepared in partnership with
%%%%  production teams to conform to Springer Nature technical requirements.
%%%%  Editorial and presentation requirements differ among journal portfolios and
%%%%  research disciplines. You may find sections in this template are irrelevant
%%%%  to your work and are empowered to omit any such section if allowed by the
%%%%  journal you intend to submit to. The submission guidelines and policies
%%%%  of the journal take precedence. A detailed User Manual is available in the
%%%%  template package for technical guidance.
%%%%%=============================================================================%%%%

\jyear{2022}%
\newgeometry{left=2cm, right=2cm}
\linespread{1.5}

%% as per the requirement new theorem styles can be included as shown below
\theoremstyle{thmstyleone}%
\newtheorem{theorem}{Theorem}%  meant for continuous numbers
%%\newtheorem{theorem}{Theorem}[section]% meant for sectionwise numbers
%% optional argument [theorem] produces theorem numbering sequence instead of independent numbers for Proposition
\newtheorem{proposition}[theorem]{Proposition}%
%%\newtheorem{proposition}{Proposition}% to get separate numbers for theorem and proposition etc.

\theoremstyle{thmstyletwo}%
\newtheorem{example}{Example}%
\newtheorem{remark}{Remark}%

\theoremstyle{thmstylethree}%
\newtheorem{definition}{Definition}%
% \usepackage{tikz}
% \newcommand*{\circled}[1]{\lower.7ex\hbox{\tikz\draw (0pt, 0pt)%
%     circle (.5em) node {\makebox[1em][c]{\small #1}};}}
\raggedbottom
%%\unnumbered% uncomment this for unnumbered level heads
%! path of figs

\graphicspath{{/Users/songshgeo/Documents/Pycharm/WGRegimes_YRB_2020/figures/}}
\begin{document}

\title[Article Title]{Identifying regime transitions for water governance at a basin scale}

%%=============================================================%%
%% Prefix	-> \pfx{Dr}
%% GivenName	-> \fnm{Joergen W.}
%% Particle	-> \spfx{van der} -> surname prefix
%% FamilyName	-> \sur{Ploeg}
%% Suffix	-> \sfx{IV}
%% NatureName	-> \tanm{Poet Laureate} -> Title after name
%% Degrees	-> \dgr{MSc, PhD}
%% \author*[1,2]{\pfx{Dr} \fnm{Joergen W.} \spfx{van der} \sur{Ploeg} \sfx{IV} \tanm{Poet Laureate}
%%                 \dgr{MSc, PhD}}\email{iauthor@gmail.com}
%%=============================================================%%

\author[1]{\fnm{Shuang} \sur{Song}}\email{songshgeo@mail.bnu.edu.cn}

\author[1]{\fnm{Shuai} \sur{Wang}}\email{shuaiwang@bnu.edu.cn}
% \equalcont{These authors contributed equally to this work.}

\author[1]{\fnm{Xutong} \sur{Wu}}\email{wuxutong@bnu.edu.cn}
% \equalcont{These authors contributed equally to this work.}
\author[2]{\fnm{Yongping} \sur{Wei}}\email{yongping.wei@uq.edu.au}
\author[3]{\fnm{Graeme S.} \sur{Cumming}}\email{graeme.cumming@jcu.edu.au}
\author[4]{\fnm{Yue} \sur{Qin}}\email{qinyue@pku.edu.cn}
\author[5]{\fnm{Xilin} \sur{Wu}}\email{wuxilin20@mails.ucas.ac.cn}
\author*[1,5]{\fnm{Bojie} \sur{Fu}}\email{bfu@rcees.ac.cn}

\affil*[1]{\orgdiv{State Key Laboratory of Earth Surface Processes and Resource Ecology}, \orgname{Beijing Normal University,}, \orgaddress{\street{Xinjiekouwai Street}, \city{Beijing}, \postcode{100875}, \state{Beijing}, \country{China}}}

\affil[2]{\orgdiv{School of Earth and Environmental Sciences}, \orgname{The University of Queensland}, \orgaddress{\city{Brisbane}, \postcode{4067}, \state{QLD}, \country{Australia}}}

\affil[3]{\orgdiv{ARC Centre of Excellence for Coral Reef Studies}, \orgname{James Cook University}, \orgaddress{\city{Townsville}, \postcode{4811}, \state{QLD}, \country{Australia}}}

\affil[4]{\orgdiv{College of Environmental Sciences and Engineering} \orgname{Peking University}, \orgaddress{\city{Beijing}, \postcode{100875}, \state{Beijing}, \country{China}}}

\affil[5]{\orgdiv{State Key Laboratory of Urban and Regional Ecology, Research Center for Eco-Environmental Sciences}, \orgname{Chinese Academy of Sciences}, \orgaddress{\city{Beijing}, \postcode{100875}, \state{Beijing}, \country{China}}}
%%==================================%%
%% sample for unstructured abstract %%
%%==================================%%

\abstract{
    Water governance determine ``who gets water, when, and how'' in most large river basins. Shifts in water governance regimes from natural to social-ecological or ``hydrosocial'' carry profound implications for human wellbeing; identifying regime changes in water governance is critical to navigating social-ecological transitions and guiding sustainability. We characterized water governance along with the three main aspects - stress, purpose, and allocation - to develop a quantitative Integrated Water Governance Index (IWGI) at a basin scale. Applying the IWGI to the rapidly-changing Yellow River Basin (YRB) in China clarifies shifts in water governance between massive supply, transformation governance, and adaptation-oriented regimes. In the YRB, the underlying causes of regime shifts were increasing water supply and demand before the governance transformation and re-allocation and regulation after the change. The IWGI offers a comprehensive and straightforward approach to linking water governance regimes to sustainability, providing valuable insights into hydrosocial transitions.
}


\keywords{
    Regime shifts,
    Water governance,
    Transformation,
    Social-hydrology,
    Sustainability
}

%%\pacs[JEL Classification]{D8, H51}
%%\pacs[MSC Classification]{35A01, 65L10, 65L12, 65L20, 65L70}

\maketitle

\section{Introduction}\label{sec1}

Water, being ``at the centre of the planetary drama of the Anthropocene'', is essential not only for earth system processes but also in supporting development and human well-being~\cite{gleeson2020a,gleeson2020b}.
As an integral part of earth system governance, successful water governance requires a deep understanding of changes in the complex relationships between humans and water~\cite{ahlstrom2021,biermann2012,steffen2020}.
Human activities stemming from our reliance on water have profoundly modified the natural water cycle, resulting in rivers that are dominated by a hybrid of social and natural drivers~\cite{sivapalan2012,qin2014a,abbott2019}.
Facing transitions from natural to human-dominated regimes, many big river basins worldwide (which are hot spots of civilization and economic growth) are urgently in need of more effective water governance~\cite{best2019,dibaldassarre2019}.

Water governance encompasses the political, social, economic, and administrative systems that regulate water use and management, dictating ``who gets water, when and how''~\cite{lasswell2018,allan2001}.
In this context, the United Nations Development Programme (UNDP) suggests that water governance determines water usage across three core aspects: ``When and what water to use?'' (stress), ``How does water provide different services for human well-being?'' (purpose), and ``Who can use water equally and efficiently?'' (allocation)~\cite{mariajacobson2013}.
% 为充分考虑这些人类活动影响,做出有助于水治理实践的评估,以往的综合指标主要可分成两类:评估水系统与评估治理系统。
Research into index-based water governance assessment generally fall into two categories: those that focus on water systems, and those that concentrate on governance systems.
On one hand, studies on governance systems typically employ an qualitative assessment to demonstrate what practices influence water governance, e.g., the OECD framework~\cite{oecd2018a}.
For quantitative studies, due to the lack of comprehensive and detailed information on key components for water governance assessment, these studies often resort to proxy datasets of human activities to create simpler indices~\cite{varis2019,huggins2020}.
On the other hand, studies focusing on water systems utilize intuitive indices to encapsulate the outcomes of governance, like the most widely concerned -water stress has been far developed by incorporating human's regulation step by step.
Specifically, traditional water stress index only demands and supply~\cite{gleick1996}, water scarcity index by involving storage~\cite{damkjaer2017}, and the even more integrated SFV-index includes flexibility~\cite{qin2019}.
Despite their usefulness, these water stress indices tend not to provide a comprehensive characterization of water governance, as they overlook the social aspect of water usage, i.e., the purpose and allocation of water use.
As a solution, we propose an integrated water governance index (IWGI) that factors in regional water use allocations and sectoral water use purpose, thereby offering a more comprehensive quantification of governance outcomes.

The impetus for developing this new index lies in the evolving practices of water governance driven by a blend of social and natural influences.
Firstly, climate change impacts on current water yield, coupled with escalating demands from economic activities and the need for water storage development, intensify water stress~\cite{qin2019,wada2014,huang2021}.
Secondly, the purpose of water in serving human well-being is witnessing a shift in trade-offs. The balance between provisioning uses (such as drinking water and food production) and non-provisioning uses (like energy production) is tilting, reflecting changes in societal needs and values~\cite{liu2017,florke2018,jaeger2019}.
Thirdly, the allocation of water across a basin is not solely determined by regional socio-economic and environmental contexts but is also increasingly influenced by systematic regulations~\cite{schmandt2021,speed2013}.
As we transition towards a human-dominated regime, these three interlinked aspects -stress, purpose, and allocation, are undergoing substantial changes.
Assessing them separately could lead to systematic failures in water governance, highlighting the need for a more integrated approach in evaluating water governance practices.

A critical step in understanding the successes and failures of water governance is to identify the different regimes that underpin it~\cite{kjellen2015, grafton2013}.
Regimes of water governance, the general guidelines of governing practices, arise within linked human-water systems (based on management, institutions, and exploitation) to create local equilibria in social-ecological structures and functions~\cite{falkenmark2021,bressers2013,loch2020,pahl-wostl2007}.
For example, under a human-dominated regime, reservoirs make water stress easier to be alleviated because of flexibility; growing energy and industrial demands make water services purposes lopsided to non-provisioning sectors; conveyance systems make water allocation more planned (Figure~\ref{fig:framework}~A)
However, the lack of a comprehensive but straightforward approach to identifying changes in water governance regimes represents a challenge for efforts to enhance the sustainability of water resource use.
Filling this gap, which is the aim of this paper, is essential for the appropriate alignment of human and water systems.


\begin{figure*}[!ht]
	\centerline{
		\includegraphics[width=1.1\linewidth]{main/framework.png}
	}
	\caption{
		\textbf{A.} Identifying the water governance regimes in transitions of a hydrosocial cycle with an integrated water governance index (IWGI). Water stress (S), purposes of water services (P), and water allocation (A) are three aspects to be considered.
		For example, reservoir construction (\ding{172} and \ding{173}) can relieve local water stress; The development of intensive irrigated agriculture (\ding{174}) and growth of energy industrial demand (\ding{175}) will change the purpose of water use; The water delivery system controls water allocation (\ding{176} and \ding{177}) within the basin system.
		\textbf{B.} Therefore, the methodology is to combine three aspects' corresponding indicators, and then an abrupt change of the IWGI can indicate a regime shift in water governance.
	}\label{fig:framework}
\end{figure*}


The Yellow River Basin (YRB), which contains the fifth-largest and most sediment-rich river in the world, needs integrated water governance because of geological and human history~\cite{mostern2021,best2019}.
Since the 1960s, governance practices such as reservoirs, levees, and conservation measures have contained the issues troubled by thousands of years of high sediment loads~\cite{wang2016a,song2020}.
However, new challenges such as decreased streamflows and water depletions occurred in more recent times, followed by different water governance practices like water use regulation and water transfer across basins~\cite{wang2019c}.
Today, it is still impossible to completely solve water stress, trade-offs between ecosystem services, or lopsided development in different regions in the YRB to the satisfaction of all actors~\cite{wohlfart2016}.
Governance challenges induced by environmental, economic, social, and political factors have resulted in YRB being among the most intensively-governed large river basins worldwide~\cite{nickum2021}.
Identifying regime shifts in water governance within the YRB can thus provide crucial insights into rapidly-changing big river basins and how governance may respond to meeting challenges to their sustainability.

% 这里我们整合了三个方向,提出了描绘流域人水关系的指数
Here, we depict three aspects of water governance -stress, purpose and allocation with corresponding indicators (see methods) and thus develop an Integrated Water Governance Index (IWGI) by equally weighting them, to indicate results from water governance (see Figure~\ref{fig:framework}~B).
% 使用案例研究
Then, by applying the index to a typical rapid-changing big river basin (the YRB), we show how IWGI helps detect and describe complicated water governance regimes comprehensively but straightforwardly.
Following synthetic analyses of the changes in water demand, supply, economic outcomes, and institutions, we interpret the leading causes of the regime shifts.
% 最后总结出一般性框架
Finally, we propose a general regime transition schema that offers a practical guideline for a coordinated approach to exploring the challenges faced by big river basin governance.


\section{Results}\label{sec2}
\subsection{Water governance regimes}
\label{Res.1}

\begin{figure}[ht!]
	\centering
	\includegraphics[width=0.9\linewidth]{main/index.jpg}
	\caption{Changes in the IWGI index. 
	\textbf{A,} Change points detection. With significant change points in 1978 and 1994, the IWGI has three different periods.
	\textbf{B,} Contributions of each dimension to the changes of IWGI within each of the three periods. Supply, purpose and allocation were respectively the main positive contributors to P1, P2 and P3.
	}
	\label{fig:IWGI}
\end{figure}

% 这一节主要展示IWGI的变化趋势和WUR的划分
With two significant breakpoints, the changes in the IWGI are divided into three periods (Figure~\ref{fig:IWGI}A) with different slopes. 
The changes are contributed by different water governance dimensions (Figure~\ref{fig:IWGI}B).
% 第一阶段
In the first period (P1, 1965-1978), the IWGI increased rapidly. 
Water supply made the most striking positive contribution (131\%), while purpose and allocation had a slight negative contribution (-11\% and -20\%).
% 第二阶段
In the second period (P2, 1979-1994), the contributions of purpose and allocation became positive and the IWGI experienced a drop because steeply declining supply capacity played a larger negative role (dropping to -188\% lower than P1). 
% 第三阶段
In the third period (P3, 1995-2013), as positive contributions from purpose (75\%) and allocation (84\%) increased further and the negative contribution of water supply lessened (-59\%), positive growth of the IWGI returned.

% 每个时期都有一个独特的、最引人注目的、对IWGI作出积极贡献的人。不同时期的三维总体特征如图所示。
Each period has a unique most striking contributor to IWGI in positive. Overall features of the three dimensions in different periods are shown in Figure~\ref{fig:phases}.
% 第一阶段到第二阶段
Throughout P1, the water governance regime was dominated by increasing supply capacities. 
% 第三阶段集中
It then experienced a shift, slowing down in increasing supply during P2, with an accompanying reverse in the contributed proportion between purpose and allocation. Finally, the contribution of all three dimensions was similar in P3 (32.91\%, 31.87\% and 35.21\% for purpose, allocation and supply respectively), making the points cluster at the centre of the diagram. 
% 总结来说,三个稳态
The three different periods corresponded to three distinct water governance regimes: a massive supply regime (P1: 1965-1978), a purpose-focused regime (P2: 1979-1993), and a many-sided governance regime (P3: 1994-2013). 


\begin{figure}[!htbp]
	\centering
	\includegraphics[width=0.9\linewidth]{main/phases.jpg}
	\caption{Combination of contributions across three dimensions in different periods (S: supply; P: purpose; A: allocation). The closer a point is to an angle of the outside triangle, the greater the proportion of the contribution of this dimension.
	The red indicator line in this plot denotes a 1:1 contributions between purpose (P) and allocation (A). When the points are below this line, the contribution ratio of allocation is lower than that of function, and \textit{vice versa}.}
	% 由于阶段一的点位于该线上方,L的净贡献比例多于P,而第二阶段的点则恰好相反。
	\label{fig:phases}
\end{figure}

\subsection{Causes of water governance regime shifts}
\label{Res.2}

\begin{figure}[th!]
	\centering
	\includegraphics[width=0.9\linewidth]{main/causes.jpg}
	\caption{
		Causes of water governance regime shifts in the Yellow River Basin: environmental change, economic growth and efficiency changes, social transformation, and water governance policies.
		\textbf{A.} Changes in total irrigated area (orange line), and water uses in per unit of area (WU/A, green dot line, see \textit{SI Appendix} Methods S2).
		\textbf{B.} Changes in gross values added (GVA) of industry and services (blue line), and their water use for unit production (WU/GVA, red dot line) respectively (\textit{SI Appendix} Methods S2).
		\textbf{C.} Completed time of each new reservoir and their surrounding region's water use percentages as a proportion of the basin's total water use (WU) at that time. Red circles denote hub reservoirs in the basin, which play a role in integrated basin water management. 
		The size of each circle indicates the magnitude of its water storage capacity. Some important reservoirs include: (1) Xiaolangdi reservoir and Sanmen Reservoir, which were constructed mainly for managing sediments; and (2) Impoundments at Liujia Gorge, Longyang Gorge, which were constructed mainly for managing flood water discharge and water supply. The named reservoirs are significant for the entire basin, not only for regional development.
		\textbf{D.} Social transformations and national-level policies related to water governance (see \textit{SI Appendix} Methods S1 and Table S2). In order, the four transformations are ``ethos of conquer nature (since 1958)'', ``reform and opening-up (since 1978)'', ``the 87 Water Allocation Scheme (since 1987)'', ``environmental regulation (since 2003)'' in order (see \textit{SI Appendix} Methods S1).
	}
	\label{fig:Causes}
\end{figure}

% 进一步挖掘IWGI变化的根本原因,灌溉区扩张和工业和服务业的经济增长是P1和P2目的变化的关键。
Digging more deeply into the underlying causes of changes in the IWGI, the expansion of irrigated area and the economic growth of industry and services were key to the change in purpose between P1 and P2. 
% P1年,黄河流域灌溉农业面积以图A的速度快速扩张,灌溉用水占主导(1965年占总用水的$81.56\%$, 1978年占总用水的$83.17\%$,图S3)。
During P1, the area of irrigated agriculture in the Yellow River Basin expanded rapidly at a rate of $0.25*10^6 ha/yr$ (Figure\ref{fig:Causes} A), and irrigation water was the dominant water use ($81.56\%$ of the total water use in 1965, and $83.17\%$ in 1978 \textit{SI Appendix} Fig. S3). 
% 然而,进入P2后,灌溉区扩张停滞,工业和服务业逐渐增长,用水需求增加(图re\ref{fig: Causes} B),导致灌溉用水量比例下降$8\%$ S3。
Entering P2, however, the expansion of irrigated area stalled and industry and services gradually took off, with more water demands (Figure\ref{fig:Causes} A and B), leading to a $8\%$ reduction in the proportion of irrigation water use (\textit{SI Appendix} S3).

% 水分利用效率由P2变化到P3。
The efficiency of water use changed from P2 to P3. 
While irrigated area resumed its expansion again in P3 (Figure\ref{fig:Causes}A), industry and urban services assumed a stronger economic role (represented by Gross Added Values, GVA) (Figure~\ref{fig:Causes}B). 
However, because of more efficient technology and better water conservation practices (\textit{SI Appendix} Fig. S4), both experienced significant declines in water use for unit irrigated area or unit production (Figure~\ref{fig:Causes}A and Figure~\ref{fig:Causes}B). 
As a result, the differences between sectors of water use were reduced while the total water consumption remained stable during P3 (\textit{SI Appendix} Fig. S3).

% 最后,环境背景、社会转型和水治理政策在这三个时期都发挥了作用。
Finally, environmental context, social transformation and water governance policies played roles in all three regimes. 
% 我们计算了每个水库的区域和盆地用水比例(R/B ratio),较高的比例代表潜在的供水作用,而不是调节作用。
We calculated the ratios of regional and basinal water use for each reservoir (R/B ratio), with a higher ratio representing a potential role for supply rather than regulation (Figure~\ref{fig:Causes}C).
% 在自然水资源相对丰富的P1年,水库大多建在需水量高的地区,水库的需水量比显著偏高。
Under the guiding ethos of ``conquering nature'', most of the reservoirs were built in regions with high water demands during P1, when natural water resources were relatively abundant (\textit{SI Appendix} Fig. S5), and R/B ratios were significantly higher (Figure~\ref{fig:Causes}C, p<0.01). 
% 在P2中,新水库的数量显著减少,水量分配受到“87调水方案”的严格控制,总库容几乎没有增加(\textit{SI  Appendix}图S6)。
In P2, the number of new reservoirs decreased significantly and allocation of water was rigorously controlled by ``the 87 Water Allocation Scheme'', with little increase in total water storage capacity (\textit{SI Appendix} Fig. S6). 
% 进入P3期后,以“环境监管”为指导的无数国家层面的水治理政策被提出,为了便于监管,新建的水库数量甚至更多,其中大部分建在R/B比较低的地区。
Entering P3, myriad national-level water governance policies were proposed under the guide of ``environmental regulation'' (Figure~\ref{fig:Causes}D), and the number of new reservoirs was even higher for facilitating and regulating objectives. Most of these were built in regions with lower R/B ratios (Figure~\ref{fig:Causes}C and \textit{SI Appendix} Fig. S6).



\section{Discussion}\label{sec12}
\subsection{Challenges along with transition of water governance regimes}

\begin{figure*}[htbp!]
	\centering
	\includegraphics[width=0.9\linewidth]{main/transition.png}
	\caption{
		Transition schema of water governance during transformation towards a hydrosocial water cycle. The natural water cycle dominates blue pathways while socio-economic feedbacks dominate red pathways. Finally, there is a transformation towards the hydrosocial water cycle where red loop increases.
		\textbf{A. Early phase.} As socio-economic systems develop, industry and services gradually demand increasing amounts of water; at the same time, increasing social organization and technological capacity allow people to manage water resources more intensively, including intensive intervention in the natural water cycle.
		\textbf{B. Late phase} With further developed and economically efficient industries and services, trade-offs between provisioning-purpose and non-provisioning water use become prominent. Rather than being determined by local socio-economic systems, water withdraws and management are scaled up to the entire basin.
		%! The note to C. makes no sense! from the refree 2.
		Thus, \textbf{C. Transformation from a natural water cycle towards the hydro-social water cycle} occurs in paralleled with a transformation towards a hydrosocial water cycle. This is generally distinguished when water resource limits are reached. The three water governance regimes seen in the YRB are identified along this transition (Regime 1: massive supply regime, Regime 2: purpose-focused regime, Regime 3: many-sided governance regime).
		\textbf{D. Water governance challenges} Through the transitional regimes, water governance faces primarily economic and environmental challenges in the early phase and social and policy challenges in the late phase.
	}
	\label{fig:summary}
\end{figure*}

% 我们为的研究结果表明,黄河流域能被识别为三个明显的稳态。
Our results show that there have been three distinct but sequential governance regimes within the YRB: a massive supply regime (P1: 1965-1978), a governance transforming regime (P2: 1979-2001) and a demand regulating regime (P3: 2002-2013). 
Furthermore, shifts between these regimes were caused by different environmental, economic, social or political drivers, then decide who gets water, when and how.
The process echoes how social systems can change water governance regimes by enhancing their adaptive capacity. %! citation
However, the IWGI quantitatively identifies and reveals that conceptual pattern, thus characterizing the transition of water governance in detail for the first time.
It is important to note that each indicator or cause changed gradually, but emerging regime shifts moves water governance along with the hydrosocial water cycle at a basin scale.
As a result, the challenges were primarily economic and environmental at the beginning of the YRB's water governance trajectory and social and policy-related towards the end (Figure~\ref{fig:summary} C and D).

% 在水资源压力较低的大规模供应制度(长江流域1965-1978年)期间,水治理实践倾向于通过建设水库和渠道来增加服务用水(当时主要是供应目的——牲畜和农作物)。
During the massive supply regime with lower water stress (1965-1978 in the YRB), water governance thus practices tended to boost water supply for services (mainly provisioning purposes then -livestock and crops) by construction of reservoirs and channels.
% 然而,正如“人类将征服自然”的流行口号所暗示的那样,水供应的增加很少考虑到人水关系不可逆转的变化,从而也急剧增加了水的需求,很少考虑到流域的保护。
As the the popular slogan ``man will conquer nature'' suggested, however, the enhancement of water supply aligned with little consideration of the irreversible changes of human-water relationship, thus also drastically increased water demand with few consideration in basinal conservation. 
\cite{zhouDecelerationChinahuman2020}.
% 在那十年里,灌溉农田和引水设施的迅速扩张,使负担过重的长江水利枢纽接近临界点,在这个临界点上,持续增加供应来满足无限的需求是不现实的
The rapid expansion of irrigated farmland and water diversion facilities in that decade brought the overburdened YRB close to the critical point, where keeping increasing supply to meet the unlimits of demand is unpractical
% 结果,自1972年以来,超过80%的地表水使用量导致了河流的频繁枯竭,随之而来的是生态问题,如湿地的萎缩和生物多样性的下降。
As a result, the over 80\% surface water use then led to depletions of the river frequently since 1972, with ecological issues, such as wetlands shrinkage and declines in biodiversity. 
% 此外,由于处于上升阶段的工业经济也受到水资源压力的限制,现有的水治理模式严重导致了巨大的社会-生态危机和对可持续性的重大挑战
In addition, as the industrial economy in the ascendant was also limited by the water stress, the existing modes of water governance led a huge social-ecological crisis and a significant challenge to sustainability rigorously
\cite{wohlfartRiverBasinCourse2016}.

The start of the governance transforming regime (P2: 1979-2001) coincided with the rising competition for water use after ``openning and reform''.
% 正如理论预测的那样
The results in the YRB keep in line with the suggestions from theorical analysis: continuous increases in water demand when the basinal total supply is stable can be accompanied by large changes in governance regime and the rapidly enhancement in overall social adaptive capacity.
% 作为制度转变的先行者
Being a pioneer in shifting governing institution, the YRB triggered a series of changes in ``who gets water, when and how'' during this regime: slowing growth of irrigated acreage; leading water-saving infrastructure; China's first water quota scheme; The preliminary cross-boundaries water transfering plan and so on
\cite{wangThirtyYearsYellow2018}.
% 因此,1999年黄河的最后一次枯竭
As a consequence, though water stress remained and increased (mainly led by reducing streamflow and flexibility), the last depletion of the Yellow River in 1999 added a footnotes to the climax of this transformation in water governance.

Finally, it came into an adaptation-oriented regime (P3: 2002-2013), when drastically shifting in societies occurred with a stable water stress.
Socio-economic trade-offs between water-dependent regions and sectors played a more important role at this regime, so water governance had to achieve efficient water allocation while balancing different purposes in the face of limited water supply
\cite{dalinBalancingwaterresource2015}.
Reconstruction of resources widespread in different industries and regions was calling for adaptation in water governance, where the widespread requirements of adjusting in the rigid quota shares from the previous regime can be an example.
% 类似的,此稳态下国家层面的水治理政策都在进行补充或调整,因为这种制度的缺失和社会的不公平正成为流域新的治理挑战。
Like this, much national-level water governing practices were proposed under the regime, because the absence of such policies with social dilemma of high quality development became new structural challenges for water governance
\cite{konarExpandingScopeFoundation2019}.

% 一般来说,在向水社会循环的转变过程中,三种治理制度之间的转变是依次发生的。
In general, shift between the governance regimes advances in parallel with the trajectory of hydrosocial cycle (Figure~\ref{fig:summary}A, B and C).
Transition from biophysical control to social and institutional control of ecosystem dynamics may become increasingly widespread in social-ecological systems as increasing anthropogenic impacts gradually change the world
\cite{bestPaceHumanInducedChange2020,cummingLinkingeconomicgrowth2018,cummingImplicationsagriculturaltransitions2014}.
With intensive anthropogenic intervention, the YRB may be the most prominent example in this general transition of water governance -``improving supply, transforming governance, and enhancing adaptation'' %! citation.
The transition regimes identified here also echo the two kinds of major water governance challenges globally (resource challenges and structural challenges, Figure~\ref{fig:summary} and \textit{SI Appendix} Fig. S9)
\cite{singhWaterGovernanceChallenges2019,porcherFacingChallengesWater2019}.
Our analysis suggests that the early phase of transition often leads to resource-focused challenges that result from economic, demographic and environmental change; while later phases are dominated by structural challenges relating to social and political aspects of governance.
Resource challenges, represented as water shortage and water supplying difficulties, are mainly faced by undeveloped and developing basins and are highly related to economic and environmental changes
\cite{allanNavigatingcomplexitiescoordinated2019,florkeWatercompetitioncities2018,liuWaterSustainabilityChina2012}.
Alternatively, highly-controlled and developed basins (especially for transboundary rivers) must mainly resolve structural challenges, such as water disputes or lack of equity, and may be in urgent need of novel flexible, efficient sociopolitical governance structures %! Kitroeff citation wrong
\cite{kitroeffThisWarCrossBorder2020,kitroeffThisWarCrossBorder2020,roobavannanRoleSectoralTransformation2017,unep-dhiTransboundaryRiverBasins2016}.
It is typical that resource challenges and structural challenges occurred sequentially during the transition of water governance within the YRB.
From the perspective of the core dimensions emphasized by the UNDP for water governance, therefore, our proposed schema connects governance challenges and the transformation of large river basins towards a hydrosocial water cycle.

\subsection{Implications, limitations, and future directions}
\label{Outlook}

% Implications
Water governance gradually changes from being a primarily local concern to becoming a national or international concern, because large river basins being critical sources of ecosystem services, economic development, and human wellbeing. 
As the ubiquitous tele-coupling are rising additional challenges of water governance in the tighter connected world, where transition between regimes aligns with different human-water relationships. 
Within the growing hydro-social cycle, the IWGI index captures the increasment of the adaptive capacity and sketches the transitional regimes of water governance, in a relatively straightforward but comprehensive way.
It is important for scientists and decision makers to recognize the changing governance challenges, because models, institutions, engineerings, and approaches developed under one regime are not necessarily useful under a different regime
%! \cite{cummingLinkingEconomicGrowth2018, reyersSocialEcologicalSystemsInsights2018}.
In the case of the YRB, an important institutions of water quota in the governance tranforming regime became rigid during the enhanced adaptation phase.
On contrary, additional water from the ambitious cross-basins water diversion project, which originated in the phase of massive supply can play an even more crucial role in today's enhancing adaptation phase.
Overall, applications of IWGI can induce a straightforward understanding of the water governance in transition of hydrosocial cycle, as descriptions from some previous conceptual frameworks suggested for all large-scale basins worldwide.

%! limitations
One of the main limitations of applying IWGI in abundant data availability for a long-term period of time. 
For an example, we used the stress-flexibility-variability (SFV index) here, one of the most comprehensive index to our knowledge, as an indicator of the water stress dimension. 
With other indicators, therefore, we obtained various datasets in the YRB: water use in different regions and sectors, streamflows in different reaches, reservoir capacity and completion time, with further more information in social, economic, and institutional for interpretation.
It means still a gap between the comprehensively identifying and widespread application of IWGI.
However, we assumed that all solutions of water governance are relative to the question of ``who gets water, when and how'', so water stress, purposes of water services, and patterns of water allocation are matters.
We suggest that choices of the indicators for the dimensions can be adapted according to available datasets as the intertwines between the underlying components are much more crucial in understanding the transition of governance regimes.

% direction
For today's world, water-related challenges remains one of the major gaps in our progress towards sustainability, while development-first strategies are still a dominant guideline in many places and may be in opposition to improving governance
\cite{xuAssessingprogresssustainable2020,liuAerosolweakenedsummermonsoons2017,greveGlobalassessmentwater2018}.
% 虽然大多数大型河流流域随着发展而在水管理技术和水利用效率方面有所改善,但淡水利用仍被认为正在接近人类-水系统可能崩溃的行星边界
Although most large river basins have shown improvements in water management technologies and water use efficiency along with development, freshwater use is still considered to be approaching planetary boundaries where human-water systems may collapse
\cite{anExploringeffectsGrain2017,degraafEnvironmentalflowlimits2019,hugginssocialecologicaldimensionschanging2020}.
% 我们的分析展示了这种路径的前方很可能是严峻的水治理挑战。
% 总的来说,这种明显的治理失败可能有两个主要原因。
% Overall, there are probably two main reasons for this apparent failure of governance.
% 首先,农业灌溉效率的显著提高通常伴随着灌溉面积的重新扩大,导致水资源压力的趋势不减(效率悖论)
% First, significant improvement in agricultural irrigation efficiency is usually accompanied by a re-expansion of irrigated area, resulting in an unabated trend of water resources stress (the paradox of efficiency)
\cite{graftonparadoxirrigationefficiency2018}.
% 其次,如果没有成功的治理,以水循环为主导的复杂治理结构可能会导致流域尺度上水资源使用的灵活性降低,破坏社会-生态系统的恢复力
In the future, without successful governance, complicated governance structures dominated by hydrosocial water cycles may result in less flexible water use and undermine the resilience of social-ecological systems at a basin scale
\cite{qinFlexibilityintensityglobal2019,leviaHomogenizationterrestrialwater2020,grillMappingworldfreeflowing2019}.
% 从这些角度来看,我们需要更好、更全面的策略来应对治理挑战,因为核心问题是复杂而难以管理的
From these perspectives, we need more flexible and comprehensive strategies to address governance challenges because the core problems are dynamic and complex
\cite{steffenemergenceevolutionEarth2020,muneepeerakulemergenceresilienceselforganized2020,bodinCollaborativeenvironmentalgovernance2017,biermannNavigatingAnthropoceneImproving2012}.
% 更深入地了解包含非线性变化、制度和过渡思想的治理,应该有助于将治理的重点转移到维持流域社会-生态系统的恢复力和提高其可持续性。
A deeper understanding of governance that incorporates ideas of non-liner change, regimes, and transitions should help to shift the focus of governance towards maintaining the resilience of the basin’s social-ecological system and improving its sustainability.


\section{Conclusion}\label{sec13}
% 围绕“谁获得水、何时获得水、如何获得水”,水治理在水社会循环转型过程中发生了三个方面的变化:水压力、水服务目的和水分配。
% 我们开发了一个综合水治理指数(IWGI),通过整合它们来检测水治理的制度转变。将IWGI应用于一个快速变化的大流域(中国黄河流域),我们解释了半个多世纪以来水治理如何在三种制度之间转变。
% 我们的方法定量地确定了长江三角洲水治理制度的一般模式,与之前具有代表性的过渡过程的理论分析一致。
% IWGI的实施将制度变迁与根本原因联系起来,为解释水治理、水社会转型和人水关系交织的变化提供了一种全面而直接的方法。
Focusing on ``who gets water, when and how'', three aspects of water governance change along with the hydrosocial cycle transition: water stress, water services purpose, and water allocation.
We developed an Integrated Water Governance Index (IWGI) to detect regime shifts in water governance by integrating them. Applying the IWGI to a rapidly-changing large river basin (the Yellow River Basin, China), we interpret how water governance shifts between three regimes over half a century.
During the massive supply regime (P1: $1965 \sim 1978$), water governance tended to boost water supply by constructing reservoirs and channels in the YRB.\
Then, the start of the governance transforming regime (P2: $1979 \sim 2001$) coincided with rising competition for water use and led to institutional changes like water-saving improvements, water quota policies, and cross-boundary water transfer plans.
Last, adaptation-oriented regime (P3: $2002 \sim 2013$) with stable high water stress resulted in trade-offs and joint regulating between water-dependent regions and sectors.
Our approach quantitatively identifies the general schema for water governance regimes in the YRB, in line with previous theoretical analysis with a representative transition process.
Linking regime shifts to the underlying causes, the implementation of IWGI offers a comprehensive and straightforward way to interpret changes in intertwines of water governance, hydrosocial transition, and human-water relationships.


\section{Methods}\label{sec11}
To develop a comprehensive and straightforward approach to identifying water governance regimes. First, we constructed the Integrated Water Governance Index (IWGI) based on three aspects (Stress, Purpose, and Allocation, see Figure~\ref{fig:framework}). Then, we analyzed the changes in the IWGI from 1965 to 2013 using change point detection methods. The normalized Indicator for each dimension affects the IWGI by changing trends and contributions.

\subsection{Integrated Water Governance Index (IWGI)}

	% 将三者合一起,即:
	As shown in the framework Figure~\ref{fig:framework}, the IWGI combines the three aspects (Stress, Purpose, and Allocation) of water governance. Each dimension keeps two directions, and we assumed the hydrosocial cycle aligns with one of them, respectively:
	\begin{equation}
		Transformation \propto S*P*A
	\end{equation}

	We selected an indicator ($I_x$, $x=S$, $P$, or $A$, corresponding to stress, purpose, and allocation, respectively) to quantify the aspects effectively. Then, the above equation was transformed into a natural logarithm to facilitate calculation:
	\begin{equation}
		Transformation \propto ln(I_S) + ln(I_P) + ln(I_A)
	\end{equation}

	Then, the Integrated Water Governance Index (IWGI) is an average of the normalized indicators $I'_x$:
	\begin{equation}
		IWGI = (I'_S + I'_P + I'_A) / 3
	\end{equation}

	where:
	\begin{equation}
		I'_x = (I_x - I_{x, min}) / (I_{x, max} - I_{x, min})
	\end{equation}

	\subsubsection*{Indicator of stress}
	We used the scarcity-flexibility-variability (SFV) water stress index proposed in Qin et al., (2019) to evaluate water stress \cite{qin2019}. This metric considers management measures (such as the construction of reservoirs) and the impact of changes in water use structure on the evaluation of water scarcity. Based on the hydrological and economic context of YRB, four second-level regions are divided (Source Region, Upper Region, Middle Region, and Lower Region, see \textit{Supporting Information S1}). For the whole YRB, the indicator of water stress $I_S$ is the average of all regions' SFV-index:
	\begin{equation}
		I_S = \frac{1}{4} * \sum_{i=1}^4 SFV_{i}
	\end{equation}

	Where $SFV_i$ is the SFV-index for region $i$, and the detailed calculation of $SFV_i$ can be found in the \textit{Supporting Information S2}.

	\subsubsection*{Indicator of purpose}
	To quantify purpose $I_P$, we used Non-Provisioning purpose Shares (NPS) of water use as an indicator. While provisioning purpose water use ($WU_{pro}$) includes domestic, irrigated, and livestock water uses, non-provisioning purpose water use ($WU_{non-pro}$) includes industrial and urban services water uses. We calculated the NPS as:
	\begin{equation}
		NPS = \frac{WU_{pro}}{WU_{pro} + WU_{non-pro}}
	\end{equation}

	In this study, we consider livestock water use, rural and urban domestic water use, and agricultural water use as provisioning water because they directly service for survival. Others are non-provisioning: services and industrial water use because they mainly service the economy.

	\subsubsection*{Indicator of allocations}
	To describe allocations $I_A$, we designed an indicator based on entropy, called Allocation Entropy Metric (AEM), which measures the degree of evenness in water allocation:

	\begin{equation}
		I_A = CEM = \sum_{i=1}^N -log(p_{i}) * p_{i}
	\end{equation}

	where $p_{i}$ is the water proportion of region $i$ to the whole basin (here, $N=4$ considering divided regions in the YRB, see \textit{Supporting Information S1}).

	\subsection{Change points detection}
		With no assumptions about the distribution of the data, we applied the Pettitt (1979) approach of change-point detection to detect a single change-point in hydrological time series with continuous data
        \cite{pettitt1979}.
		It tests $H0$: The variables follow one or more distributions with the exact location parameter (no change) against the alternative: a change point exists.
		Mathematically, when a sequence of random variables is divided into two segments represented by $\mathrm{x}_{1}, \mathrm{x}_{1}, \ldots, x_{t_{0}}$ and $x_{t_{0}+1}, x_{t_{0}+2}, \ldots, x_{T}$, if each segment has a common distribution function, i.e., $F_1(x)$, $F_2(x)$ and $F_1(x) \neq F_2(x)$, then the change point is identified at $t_0$. To achieve the identification of change point, a statistical index $U_{t,T}$ is defined as follows:

		\begin{equation}
			U_{t, T} = \sum_{i=1}^t\sum_{j=t+1}^T sgn(X_i - X_j), 1 \leq t < T
		\end{equation}

		where:
		\begin{equation}
			\operatorname{sgn}(\theta)= \begin{cases}1 & \text { if } \theta>0 \\ 0 & \text { if } \theta=0 \\ -1 & \text { if } \theta<0\end{cases}
		\end{equation}

		The most probable change point $\tau$ is found where its value satisfies $K_{\tau} = max|U_{t, T}|$ and the significance probability associated with value $K_{\tau}$ is approximately evaluated as:
		\begin{equation}
			p=2 \exp \left(\frac{-6 K_{\tau}^{2}}{T^{2}+T^{3}}\right)
		\end{equation}
		Given a certain significance level $\alpha$, if $p < \alpha$, we reject the null hypothesis and conclude that $x_{\tau}$ is a significant change point at level $\alpha$.

		%实际上这个阈值取0.001至0.1不会影响我们的结果,我们识别的断点是相当稳定的
		We used $\alpha = 0.001$ as the threshold level of the p-value, meaning that the probability of a statistically significant change-point judgment being valid was more than $99.9\%$. We divided the series into two at that point and analyzed each series separately until all significant change points were detected. Though two break points in the main text with $\alpha = 0.001$, the threshold from $0.0005$ to $0.05$ does not affect our results, and the breakpoints we identified are robust (see \\textit{Supporting Information} Figure~S3).

	\subsection{Datasets}
	% 为了计算 IWGI ,我们需要计算多个指标及子指标,所有使用的数据集都在表中列出,数据的详细介绍可见补充材料。
	In order to calculate IWGI in the YRB, all the datasets we need are listed in the \textit{SI Appendix} Table~S1 with a detailed description in \textit{Supporting Information S3}.


\backmatter

\bmhead{Supplementary information}
When calculating the indicators (especially the SFV water stress index and the allocation entropy metric), we should use data at a lower (regional scale in this study) spatial scale. Therefore, we divide the YRB into four regions: source region (SR), upper region (UR), middle region (MR), and lower region (LR), according to characteristics and customary practices in the \ref{secA1}.
The formulation in detail for applying the SFV-index is available in the \ref{secA2}.
We used multiple sources of datasets in this study, \textit{Appendix}~\ref{secA3} introduces where they came from and how we harmonise them for analysis.

\bmhead{Acknowledgments}

Funding was provided by the National Natural Science Foundation of China (CN) (Grant Nos. NSFC 42041007).


\section*{Declarations}

% Some journals require declarations to be submitted in a standardised format. Please check the Instructions for Authors of the journal to which you are submitting to see if you need to complete this section. If yes, your manuscript must contain the following sections under the heading `Declarations':

\begin{itemize}
\item Funding
\item Conflict of interest/Competing interests (check journal-specific guidelines for which heading to use)
\item Ethics approval
\item Consent to participate
\item Consent for publication
\item Availability of data and materials
\item Code availability
\item Authors' contributions
\end{itemize}

\noindent
If any of the sections are not relevant to your manuscript, please include the heading and write `Not applicable' for that section.

%%===================================================%%
%% For presentation purpose, we have included        %%
%% \bigskip command. please ignore this.             %%
%%===================================================%%
\bigskip
\begin{flushleft}%
Editorial Policies for:

\bigskip\noindent
Springer journals and proceedings: \url{https://www.springer.com/gp/editorial-policies}

\bigskip\noindent
Nature Portfolio journals: \url{https://www.nature.com/nature-research/editorial-policies}

\bigskip\noindent
\textit{Scientific Reports}: \url{https://www.nature.com/srep/journal-policies/editorial-policies}

\bigskip\noindent
BMC journals: \url{https://www.biomedcentral.com/getpublished/editorial-policies}
\end{flushleft}

\begin{appendices}

\section{YRB Regions}\label{secA1}
We divide the YRB into four regions to calculate the indicators considering both socio-economic and natural conditions. The division aligns with the customary schema from publications and the YRCC~\cite{yellowriverconservancycommission2013,wang2019c,wang2016e}, so four important hydrological stations can distinguish the regions (see Figure~\ref{fig:YRB}).

\begin{itemize}
    \item \textbf{Source Region (SR):} Over 50\% of natural runoff originates from this region. The most ecological function here is water yield, as sparsely populated and less economically developed.
    \item \textbf{Upper Region (UR):} With the highest per capita irrigated land area, there are numbers of large irrigation lands in this region. However, irrigation efficiency is relatively much lower than its lower reaches.
    \item \textbf{Middle Region (MR):} Crossing Loess Plateau, a famous rich-sand area, Yellow River loads most of its sediments here with the highest soil erosion risk. The ``grain for the green'' project changed the water utilization here strikingly to reverse this situation~\cite{wu2020a}.
    \item \textbf{Lower Region (LR):} With a dense population and the traditional agricultural trajectory, the lower region used to be the largest water use region. However, as the industrial transformation going, the proportion of agriculture keeps decreasing, but LR is still the largest water use region in each aspect.
\end{itemize}

% 补充图片1:研究区示意图
\begin{figure*}[hbtp!]
    \centering
    \includegraphics[width=0.6\textwidth]{sup/s1_study_area.jpg}
    \caption{
        The study area.
        \textbf{A.} Diagram of the YRB and the subdivision of the basin (SR: Source Region, UR: Upper Region, MR: Middle Region, DR: Downstream region).
        \textbf{B.} Profile of the main channel of the Yellow River. The hydrological stations control the SR, UR, MR and DR.
        \textbf{C.} Typical landscapes in different regions in the YRB.
    }\label{fig:YRB}
\end{figure*}

% % 补充图片4:天然水资源量
% \begin{figure*}[tb]
%     \centering
%     \includegraphics[width=0.8\textwidth]{sup/sf_measured_runoff.pdf}
%     \caption{Water resources in different regions.
%         \textbf{A,} changing trend of measured runoff,
%         \textbf{B, C and D} average measured runoff within different periods.
%         \textbf{E, F and G} average total water consumptions within different periods.
%     }
%     \label{fig:water resources}
% \end{figure*}


\section{SFV-index}\label{secA2}
By taking water flexibility and variability into account, the scarcity-flexibility-variability (SFV) index focus more on dynamic responses to water resources in a developing perspective, which is a valid metric of temporal changes in water stresses \cite{qin2019}. To apply this method, we need to combine three metrics following:

First, for scarcity, $A_{i, j}$ is the total water consumption as a proportion of regional multi-year average runoff volume in year $j$ and region $i$ (in this study, four regions in the YRB, \textit{Appendix}~\ref{secA1}):
\begin{equation}
    A_{i, j} = \frac{WU_{i,j}}{R_{i, avg}}
\end{equation}

Second, for flexibility, $B_{i, j}$ is the inflexible water use $WU_{inflexible}$ (i.e. for thermal power plants or humans and livestock) as a proportion of average multi-year runoff, in year $i$ and region $j$:
\begin{equation}
    B_{i, j} = \frac{WU_{i, j, inflexible}}{R_{i, avg}}
\end{equation}

Finally for variability, the capacity of the reservoir and the positive effects of storage on natural runoff fluctuations are also considered.
\begin{gather}
C_i = C1_i * (1 - C2_i) \\
C1_{i, j} = \frac{R_{i, std}}{R_{i, avg}} \\
C2_{i} = \frac{RC_{i}}{R_{i, avg}}, \ if RC < R_{i, avg} \\
C2_{i} = 1, \ if RC >= R_{i, avg}
\end{gather}

In all the equations above, $R_{i, avg}$ is the average runoff in region $i$, $RC_i$ is the total storage capacities of reservoirs in the region $i$, $R_{i, std}$ is the standard deviation of runoff in the region $i$.

Finally, assuming three metrics (scarcity, flexibility and variability) have the same weights, we can calculate the $SFV$ index after normalizing them:
\begin{gather}
    V = \frac{A_{normalize} + B_{normalize} + C_{normalize}}{3}\\
    a = \frac{1}{V_{max} - V_{min}};\\
    b = \frac{1}{V_{min} - V_{max}} * V_{min}\\
    SFV = a * V + b
\end{gather}



% \begin{figure}[tb!]
% 	\centering
% 	\includegraphics[width=0.9\linewidth]{sup/S3_wci.pdf}
% 	\caption{
% 	    Ratio of irrigated areas with water-saving equipments.
% 	}
% 	\label{fig:wci}
% \end{figure}


\begin{figure}[tb]
	\centering
	\includegraphics[width=0.6\linewidth]{sup/reservoirs.pdf}
	\caption{
	    Numbers of new reservoirs in each year.
	}
	\label{fig:reservoirs}
\end{figure}

\begin{figure}
    \centering
    \includegraphics[width=\linewidth]{sup/sensitivity.pdf}
    \caption{
        Sensitivity analysis of the threshold of p-values.
        \textbf{A.} number of breakpoints in different p-values, the scheme with two-breakpoints are the dominant situation.
        \textbf{B.} Threshold of p-values $\alpha=0.0005$.
        \textbf{C.} Threshold of p-values $\alpha=0.05$.
    }
    \label{fig:sensitivity}
\end{figure}


\section{Datasets}\label{secA3}
\subsection*{Descriptions}
This study used multiple types of data (see Table~\ref{tab:datasets}): statistical datasets, hydrological datasets, and political datasets.

\subsubsection*{Statistical datasets}
% We used GDP data; water resources uses data extracted from the 2nd National Water Resources Assessment Program \cite{zhou2020} and statistical yearbooks \url{http://www.yrcc.gov.cn/other/hhgb/}.
The water resources use dataset was published by Zhou et al. \cite{zhou2020}, which records water utilization in different sectors along with social-economic situations at the Prefectures level. 2nd National Water Resources Assessment Program mainly extracted this dataset launched in 2002, led by the National Development and Reform Commission and the Ministry of Water Resources (see ref (1) and \url{http://www.mwr.gov.cn/english/publs/} for more details). Since then, the statistics from the survey using the same criteria have been supplemented and harmonized with the 2013 administrative divisions.

The data covers a total of subcategories of water use under four broad categories: agriculture (IRR), industry (IND), urban (URB) and rural (RUR) water use (see Zhou et al., for details \cite{zhou2020}).
% There is uncertainty at the county scale for each disaggregated water use sector, but because the corrected data has been for statistical information using the water balance method, the data are adequate for the regional scale used in this study.

\subsubsection*{Hydrological datasets}
The reservoir dataset was collected by Wang et al. \cite{wang2019c}, which introduced includes the significant new reservoirs built in the YRB since 1949 (Figure~\ref{fig:reservoirs}). YRCC labelled the regulation-oriented reservoirs among them, see \url{http://www.yrcc.gov.cn/hhyl/sngc/}). In addition, annual runoff data derived from hydrological station measurements are the same as the datasets used in \cite{wang2019c} and \cite{wang2016e}.

\subsubsection*{Political datasets}
The policy dataset collects laws and policies listed in the book \cite{yellowriverconservancycommission2013}, which are related to the Yellow River basin promulgated and implemented by departments at (such as YRCC) and above (such as national institutions) at the Basin's level (Table~\ref{tab:policies}).
In addition, some are difficult to categorize; not a landmark, but numerous water governance practices in the YRB had been recorded in ``Yellow River Events'' by the YRCC; we collected them from \url{http://www.yrcc.gov.cn/hhyl/hhjs/}.

\subsection*{Methods S3. Harmonization}
Due to the wide sources of our data set and the different spatial scales, we need to harmonize them into a practical scale.
\begin{itemize}
    \item 1. Datasets at watersheds scales:
        We directly divided the annual hydrological data and measured runoff data according to their watersheds' corresponding hydrological stations (see Figure~\ref{fig:YRB} A and B).
    \item 2. Prefecture:
        We calculate the area of each prefecture to determine whether they belong to a region, with the threshold of $95\%$:

        \begin{equation}
            S_{ij} = MAX(S_ij / S_i)
        \end{equation}

        Where $i$ refers to a specific prefecture and $j$ refers to a region within YRB, i.e. SR, UR, MR, or DR. $S_i$ refers to the area of perfect $i$, and $S_{ij}$ refers intersecting area between perfect $i$ and region $j$.
        We define perfecture $i$ belongs to region $j$ if their intersecting area $S_{ij}$ over 95\% of $S_i$, i.e.:

        \begin{equation}
            MAX(S_{ij}) > 0.95 * S_i
        \end{equation}
    \item 3. Province:
        According to the major provinces contained in different regions, we determine which region the data of that province is merged into by referring to the traditional division practice:
    \begin{itemize}
        \item SR: Qinghai Gansu and Sichuan,
        \item UR: Ningxia and Inner Mongolia,
        \item MR: Shanxi and Shaanxi,
        \item DR: Shandong, Hebei and Henan.
    \end{itemize}
\end{itemize}

Finally, when we process the location data (i.e., the location data of the reservoirs), we judge the province it belongs to according to its location and then fit it to the regional scale.

\begin{table*}[!htbp]
    \begin{center}
    \small
    \begin{minipage}{\linewidth}
    \caption{Used datasets and their sources.}\label{tab:datasets}%
    \resizebox{\linewidth}{!}{
    \begin{tabular}{lrrrp{0.4\linewidth}}
    \toprule
    Dataset & Type & Spatial scale & Time scale & Source \\
    \midrule
    1. Administrative water use & Statistical & Prefectures & 1965-2013 & 2nd National Water Resources Assessment Program \cite{zhou2020}\\
    2. GDP & Statistical & Province & 1949-2019 & Wind database \\
    3. Streamflow withdrawals & Statistical & Watershed & 2003-2019 & Yearbooks \url{http://www.yrcc.gov.cn/other/hhgb/} \\
    4. Reservoirs & Hydrological & Location & 1949-2015 & Publication \cite{wang2019c} \\
    5. Measured runoff & Hydrological & Location & 1949-2019 & Measured data \cite{wang2019c,wang2016e} \\
    6. Laws & Political & Documents & 1949-2013 & YRCC \cite{yellowriverconservancycommission2013} \\
    7. History of YRCC & Political & Documents & 1949-2002 & YRCC \cite{yellowriverarchives2004} \\
    8. YRB Events & Political & Documents & 1949-2015 & YRCC: \url{http://www.yrcc.gov.cn/hhyl/hhjs/} \\
    \botrule
    \end{tabular}}
    \end{minipage}
    \end{center}
    \end{table*}

    %% 表格
\begin{table*}
    \centering
    \small
    \caption{Policies and regulations above YRB level which affected the whole basin in water utilization}\label{tab:policies}
    \begin{minipage}{\linewidth}
    \resizebox{\linewidth}{!}{
    \begin{tabular}{p{0.6\linewidth}rp{0.35\linewidth}}
        Name & Year & Agency \\
        \midrule
        1. Water Law of PRC & 1988 & National People's Congress of the PRC \\
        2. Water Law of PRC -revised 1 & 2009 & National People's Congress of the PRC \\
        3. Water Law of PRC -revised 2 & 2016 & National People's Congress of the PRC \\
        4. Regulations on the Administration of Water Drawing Licences and The Collection of water resource fees & 2006 & State Council of the PRC \\
        5. Regulations on the Administration of Water Drawing Licences and The Collection of water resource fees -revised 1 & 2017 & State Council of the PRC \\
        6. Regulations on the Allocation of Water in the Yellow River & 2006 & State Council of the PRC \\
        7. Yellow River water supply distribution scheme & 1987 & State Council of the PRC \\
        8. Measures for the Administration of Water Drawing Permits & 2008 & Ministry of Water Resources of the PRC \\
        9. Measures for the Administration of Water Drawing Permits -revised 1 & 2015 & Ministry of Water Resources of the PRC \\
        10. Measures for the Administration of Water Drawing Permits -revised 2 & 2017 & Ministry of Water Resources of the PRC \\
        11. Regulations on the Allocation of Water in the Yellow River & 2006 & State Council of the PRC \\
        12. Annual distribution of available water supply of the Yellow River and mainstream water dispatching scheme & 1998 & Ministry of Water Resources of the PRC \\
        13. The Yellow River water dispatching management measures & 1998 & Ministry of Water Resources \\
        14. Measures for the Implementation of the Yellow River Water Rights Conversion Management & 2004 & Ministry of Water Resources \\
        15. Regulations on the Administration of Water Drawing Licences and The Collection of water resource fees & 2006 & State Council of the PRC \\
        16. Measures for the implementation of the water drawing Permit system & 1993 \\ State Council of the PRC \\
        17. Measures for the demonstration and management of water resources in construction projects & 2002 & Ministry of Water Resources of the PRC \\
        18. Implementation Opinions on the Reform of Water Conservancy Project Management System & 2006 & State Council of the PRC \\
        \bottomrule
    \end{tabular}}

    \footnotesize[1]{If a policy was proposed by multiple legacies, we only show the highest one.}
    \end{minipage}
\end{table*}


\end{appendices}

%%===========================================================================================%%
%% If you are submitting to one of the Nature Portfolio journals, using the eJP submission   %%
%% system, please include the references within the manuscript file itself. You may do this  %%
%% by copying the reference list from your .bbl file, paste it into the main manuscript .tex %%
%% file, and delete the associated \verb+\bibliography+ commands.                            %%
%%===========================================================================================%%

\bibliographystyle{sn-standardnature}
\bibliography{mybib}% common bib file
%% if required, the content of .bbl file can be included here once bbl is generated
%%\input sn-article.bbl

%% Default %%
%%\input sn-sample-bib.tex%

\end{document}
