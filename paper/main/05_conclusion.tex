
Three dimensions of water governance change along with the hydrosocial cycle transition: water stress, services purpose, and water allocation, affecting ``who gets water, when and how''. We developed an Integrated Water Governance Index (IWGI) to detect regime shifts in water governance by integrating them. Applying the index to a rapidly-changing large river basin (the Yellow River Basin, China) describes how water governance shifts between three regimes over half a century (massive supply regime; governance transformation regime; and adaptation oriented regime, respectively). Our approach quantitatively identifies the general schema for water governance regimes in the YRB, in line with previous theoretical analysis with a representative transition process. Linking regime shifts to the underlying causes, the implementation of IWGI offers a comprehensive and straightforward way to interpret changes in intertwines of water governance, hydrosocial transition, and human-water relationships.
