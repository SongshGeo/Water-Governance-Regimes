% 围绕“谁获得水、何时获得水、如何获得水”,水治理在水社会循环转型过程中发生了三个方面的变化:水压力、水服务目的和水分配。
% 我们开发了一个综合水治理指数(IWGI),通过整合它们来检测水治理的制度转变。将IWGI应用于一个快速变化的大流域(中国黄河流域),我们解释了半个多世纪以来水治理如何在三种制度之间转变。
% 我们的方法定量地确定了长江三角洲水治理制度的一般模式,与之前具有代表性的过渡过程的理论分析一致。
% IWGI的实施将制度变迁与根本原因联系起来,为解释水治理、水社会转型和人水关系交织的变化提供了一种全面而直接的方法。
Focusing on ``who gets water, when and how'', three aspects of water governance change along with the hydrosocial cycle transition: water stress, water services purpose, and water allocation.
We developed an Integrated Water Governance Index (IWGI) to detect regime shifts in water governance by integrating them. Applying the IWGI to a rapidly-changing large river basin (the Yellow River Basin, China), we interpret how water governance shifts between three regimes over half a century.
During the massive supply regime (P1: $1965 \sim 1978$), water governance tended to boost water supply by constructing reservoirs and channels in the YRB.\
Then, the start of the governance transforming regime (P2: $1979 \sim 2001$) coincided with rising competition for water use and led to institutional changes like water-saving improvements, water quota policies, and cross-boundary water transfer plans.
Last, adaptation-oriented regime (P3: $2002 \sim 2013$) with stable high water stress resulted in trade-offs and joint regulating between water-dependent regions and sectors.
Our approach quantitatively identifies the general schema for water governance regimes in the YRB, in line with previous theoretical analysis with a representative transition process.
Linking regime shifts to the underlying causes, the implementation of IWGI offers a comprehensive and straightforward way to interpret changes in intertwines of water governance, hydrosocial transition, and human-water relationships.
