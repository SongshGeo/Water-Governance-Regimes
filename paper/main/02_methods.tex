% chktex-file 46

To develop a comprehensive and straightforward approach to identifying water governance regimes. First, we constructed the Integrated Water Governance Index (IWGI) based on three aspects (Stress, Purpose, and Allocation, see Figure~\ref{fig:framework}). Then, we analyzed the changes in the IWGI from 1965 to 2013 using change point detection methods. The normalized indicator for each dimension affects the IWGI by changing trends and contributions.

\subsection{Integrated Water Governance Index (IWGI)}

	% 将三者合一起,即:
	As shown in the framework Figure~\ref{fig:framework}, the IWGI combines the three aspects (Stress, Purpose, and Allocation) of water governance:
	\begin{equation}
		Transformation \propto S*P*A
	\end{equation}

	We selected an indicator ($I_x$, $x=S$, $P$, or $A$, corresponding to stress, purpose, and allocation, respectively) to quantify the aspects effectively. Then, the above equation was transformed into a natural logarithm to facilitate calculation:
	\begin{equation}
		Transformation \propto \ln(I_S) + \ln(I_P) + \ln(I_A)
	\end{equation}

	Then, the Integrated Water Governance Index (IWGI) is an average of the normalized indicators $I'_x$:
	\begin{equation}
		IWGI = (I'_S + I'_P + I'_A) / 3
	\end{equation}

	where $I'_x$ is calculated by Min-Max normalization of $I_x$ (thus ranges from zero to one):
	\begin{equation}
		I'_x = (I_x - I_{x, \min}) / (I_{x, \max} - I_{x, \min})
	\end{equation}

	Since the IWGI essentially comprises by three aspects' indicator with same weights, its prerequisite is to keep the same data source for each indicator throughout time, to ensure time series continuity.
	However, vary data sources can be used when estimating the specific indicator or cross different indicators, which makes IWGI a flexible framework for substituted indicators.

	\subsubsection{Indicator of stress (IS)}
	We used the scarcity-flexibility-variability (SFV) water stress index proposed by \citeA{qin2019} to evaluate water stress.
	This indicator integrates the share of runoff being consumed, the share of consumption in these inflexible categories and the historical variability of runoff weighted by storage capacity~\cite{qin2019}, where impacts from both management measures and climate changes are included.
	% 这个指数得到了许多应用,是就我们已知范围内考虑因素最为齐全的,反应水资源压力的指数。
	The SFV-index, which has many applications, is the most comprehensive index of water stress we know. % todo citation

	Based on the hydrological and economic context of YRB, four second-level regions are divided (Source Region, Upper Region, Middle Region, and Lower Region, see \textit{Supporting Information Section~S1}).
	For the whole YRB, the indicator of water stress $I_S$ is the average of all regions' SFV-index:
	\begin{equation}
		I_S = \frac{1}{4} * \sum_{i=1}^4 SFV_{i}
	\end{equation}

	Where $SFV_i$ is the scarcity-flexibility-variability (SFV) index of region $i$. By taking water flexibility and variability into account, the SFV focus more on dynamic responses to water resources in a developing perspective, which is a valid metric of temporal changes in water stresses~\cite{qin2019}. To apply this method, we need to combine three metrics: scarcity, flexibility and variability.
	In all the equations following, $R_{i, avg}$ is the average runoff in region $i$, $RC_i$ is the total storage capacities of reservoirs in the region $i$, $R_{i, std}$ is the standard deviation of runoff in the region $i$.

	First, for scarcity, $A_{i, j}$ is the total water consumption as a proportion of regional multi-year average runoff volume in year $j$ and region $i$ (in this study, four regions in the YRB, \textit{Supporting Information Section S1}):
	\begin{equation}
		A_{i, j} = \frac{WU_{i,j}}{R_{i, avg}}
	\end{equation}

	Second, for flexibility, $B_{i, j}$ is the inflexible water use $WU_{inflexible}$ (i.e.\ for thermal power plants or humans and livestock) as a proportion of average multi-year runoff, in year $i$ and region $j$:
	\begin{equation}
		B_{i, j} = \frac{WU_{i, j, inflexible}}{R_{i, avg}}
	\end{equation}

	Finally for variability, the capacity of the reservoir and the positive effects of storage on natural runoff fluctuations are also considered.
	\begin{gather}
	C_i = C1_i * (1 - C2_i) \\
	C1_{i, j} = \frac{R_{i, std}}{R_{i, avg}} \\
	C2_{i} = \frac{RC_{i}}{R_{i, avg}}, \ if RC < R_{i, avg} \\
	C2_{i} = 1, \ if RC >= R_{i, avg}
	\end{gather}

	Finally, assuming three metrics (scarcity, flexibility and variability) have the same weights, we can calculate the $SFV$ index after normalizing them:
	\begin{gather}
		V = \frac{A_{normalize} + B_{normalize} + C_{normalize}}{3}\\
		a = \frac{1}{V_{\max} - V_{\min}};\\
		b = \frac{1}{V_{\min} - V_{\max}} * V_{\min}\\
		SFV = a * V + b
	\end{gather}

	\subsubsection{Indicator of purpose (IP)}
	To quantify purpose $I_P$, we used provisioning purpose shares (PPS) of water use as an indicator. While provisioning purpose water use ($WU_{pro}$) includes domestic, irrigated, and livestock water uses, non-provisioning purpose water use ($WU_{non-pro}$) includes industrial and urban services water uses. We calculated the PPS by:
	\begin{equation}
		PPS = \frac{WU_{pro}}{WU_{pro} + WU_{non-pro}}
	\end{equation}

	In this study, we consider livestock water use, rural and urban domestic water use, and agricultural water use as provisioning water because they directly service for survival. Others are non-provisioning: services and industrial water use because they mainly service the economy.

	\subsubsection{Indicator of allocations (IA)}
	To describe allocations $I_A$, we designed an indicator based on entropy, called Allocation Entropy Metric (AEM), which measures the degree of evenness in water allocation:

	\begin{equation}
		I_A = AEM = \sum_{i=1}^N - \log(p_{i}) * p_{i}
	\end{equation}

	where $p_{i}$ is the proportion of regional water use in $i$ to water use of the whole basin (here, $N=4$ considering divided regions in the YRB, see \textit{Supporting Information S1}).

	\subsection{Change points detection}
		We applied the Pettitt (1979) approach to detect change-points of IWGI within continuous data, since this method has no assumptions about the distribution of the data~\cite{pettitt1979}.
		It tests $H0$: The variables follow one or more distributions with the exact location parameter (no change) against the alternative: a change point exists.
		Mathematically, when a sequence of random variables is divided into two segments represented by $\mathrm{x}_{1}, \mathrm{x}_{1}, \ldots, x_{t_{0}}$ and $x_{t_{0}+1}, x_{t_{0}+2}, \ldots, x_{T}$, if each segment has a common distribution function, i.e., $F_1(x)$, $F_2(x)$ and $F_1(x) \neq F_2(x)$, then the change point is identified at $t_0$. To achieve the identification of change point, a statistical index $U_{t,T}$ is defined as follows:

		\begin{equation}
			U_{t, T} = \sum_{i=1}^t\sum_{j=t+1}^T sgn(X_i - X_j), 1 \leq t < T
		\end{equation}

		where:
		\begin{equation}
			\operatorname{sgn}(\theta)= \begin{cases}1 & \text { if } \theta>0 \\ 0 & \text { if } \theta=0 \\ -1 & \text { if } \theta<0\end{cases}
		\end{equation}

		The most probable change point $\tau$ is found where its value satisfies $K_{\tau} = \max|U_{t, T}|$ and the significance probability associated with value $K_{\tau}$ is approximately evaluated as:
		\begin{equation}
			p=2 \exp \left(\frac{-6 K_{\tau}^{2}}{T^{2}+T^{3}}\right)
		\end{equation}
		Given a certain significance level $\alpha$, if $p < \alpha$, we reject the null hypothesis and conclude that $x_{\tau}$ is a significant change point at level $\alpha$.

		%实际上这个阈值取0.001至0.1不会影响我们的结果,我们识别的断点是相当稳定的
		We used $\alpha = 0.001$ as the threshold level of the p-value, meaning that the probability of a statistically significant change-point judgment being valid was more than $99.9\%$. We divided the series into two at that point and analyzed each series separately until all significant change points were detected. Though two break points in the main text with $\alpha = 0.001$, the threshold from $0.0005$ to $0.05$ does not affect our results, and the breakpoints we identified are robust (see Figure~S6).

	\subsection{Datasets}\label{sec:datasets}

	For calculating IWGI, three datasets were used: reservoirs, measured runoff, and water uses.
	The reservoir dataset was collected by~\citeA{wang2019c}, which introduced includes the significant new reservoirs built in the YRB since 1949 (Accessible with reasonable request).
	Among all the reservoirs, YRCC labelled the ``major reservoirs'' which were constructed mainly for regulating and managing (see \url{http://www.yrcc.gov.cn/hhyl/sngc/}).
	In addition, annual measured runoff data was collected from the Yellow River Sediment Bulletin (\url{http://www.yrcc.gov.cn/nishagonggao/}) and four controlling stations are measuring different reaches of the Yellow River (see Supporting Informations Section~S1).
	The water resources use dataset was from National Long-term Water Use Dataset of China (NLWUD) published by \citeA{zhou2020}, which includes water uses, water-consuming economic variables, and water use intensities by sectors the prefectures level.
	We determined the prefectures belong to the YRB by filtering the NLWUD dataset with a threshold of $95\%$ intersected area.

	For analyzing its causes of changing water, irrigated area, gross added values of industry and services, and water use intensities data were also from NLWUD dataset~\cite{zhou2020}.
	Besides, two water governance policies datasets are used: laws data and ``big events'' documents dataset.
	Data of laws were collected from~\citeA{yellowriverwaterconservancycommission2010}, which reviewed all important laws at the basin scale related to the Yellow River from the last century.
	The original documents of ``big events'' related to the Yellow River come from the YRCC, the agency at the basin scale, which recorded and compiled these events (\url{http://www.yrcc.gov.cn/hhyl/hhjs/}).

	We calculated the IWGI from 2001 to 2017 (the latest) in the Supporting Information Section S4 for robustness test with another water use dataset from Yellow River Water Resources Bulletin (\url{http://www.yrcc.gov.cn/zwzc/gzgb/gb/szygb/}).
