This analysis aimed to develop a comprehensive and straightforward approach to identifying water governance regimes. First, we constructed the Integrated Water Governance Index (IWGI) based on three dimensions (Stress, Purpose, and Allocation, see Figure~\ref{fig:framework}). Then, we analyzed the changes in the IWGI from 1965 to 2013 by using change point detection methods. The normalized indicator for each dimension effects the IWGI by both changing trend and contributions.

\subsection{Integrated Water Governance Index (IWGI)}

	% 将三者合一起,即:
	As shown in the framework Figure~\ref{fig:framework}, the IWGI combines the three core dimensions (Stress, Purpose, and Allocation) of water governance. Each dimension keeps two directions and we assumed the hydrosocial cycle aligns with one of them, respectively:
	\begin{equation}
		Transformation \propto S*P*A
	\end{equation}

	To effectively quantify the dimensions, we selected an indicator ($I_x$, $x=S$, $P$ or $A$, corresponding to stress, purpose, and allocation respectively) for each dimension. Then, the above equation was transformed into a natural logarithm to facilitate calculation:
	\begin{equation}
		Transformation \propto ln(I_S) + ln(I_P) + ln(I_A)
	\end{equation}

	Then, the Integrated Water Governance Index (IWGI) is average of the normalized indicators $I'_x$:
	\begin{equation}
		IWGI = I'_S + I'_P - I'_A
	\end{equation}

	where:
	\begin{equation}
		I'_x = (I_x - I_{x, min}) / (I_{x, max} - I_{x, min})
	\end{equation}

	\subsubsection*{Indicator of stress}
	We used the scarcity-flexibility-variability (SFV) water stress index proposed in Qin et al., 2019 to evaluate water stress ($SFV_i$) as the indicator in a particular region $i$ \cite{qinFlexibilityintensityglobal2019}. This metric takes into account management measures (such as the construction of reservoirs) and the impact of changes in the industrial structure of water use on the evaluation of water scarcity (see \textit{SI Appendix} Methods S4 for details). For the whole YRB, the indicator of water stress $I_S$ is the average of all regions' SFV-index:
	\begin{equation}
		I_S = \frac{1}{4} * \sum_{i=1}^4 SFV_{i}
	\end{equation}

	Where $SFV_i$ is the SFV-index for region $i$, and $i=1$ to $4$ refers SR, UR, MR, and DR (see \textit{SI Appendix} Methods S1 Definition of study area).

	\subsubsection*{Indicator of purpose}
	To quantify purpose $I_P$, we used Non-Provisioning purpose Shares (NPS) of water use as an indicator. While provisioning purpose water use ($WU_{pro}$) includes domestic, irrigated and livestock water uses, non-provisioning purpose water use ($WU_{non-pro}$) includes industrial and urban services water uses. We calculated the NPS as:
	\begin{equation}
		NPS = \frac{WU_{pro}}{WU_{pro} + WU_{non-pro}}
	\end{equation}

	\subsubsection*{Indicator of allocations}
	%$AEM$ 指标度量的是水资源配置“不均匀”的程度,类似于信息熵是对“混乱程度”的度量,即参与分配的各个单位之间,分配比例差距越大,则熵越小。
	% 而我们的指标应反映随着社会发展,水资源配置在区域之间更加均衡(比例差距小)、整体满足不同用水部门的发展需求(比例差距减小)、但不同区域存在部门分工(比例差距增大)的趋势。
	To describe allocations $I_A$, we designed an indicator based on entropy, called Allocation Entropy Metric (AEM), which measures the degree of evenness in water allocation:

	\begin{equation}
		I_A = CEM = \sum_{i=1}^N -log(p_{i}) * p_{i}
	\end{equation}

	where $p_{i}$ is the water proportion of region $i$ to the whole basin ($N$ regions in sum). The YRB is divided into 4 regions (i.e., $N = 4$) according to the natural and socio-economic backgrounds (see \textit{SI Appendix XX}).

	\subsection{Change points detection}
		% The method makes no assumptions about the distribution of the data and detects breakpoints based solely on the probability of the data coming from different distributions before and after the breakpoint.

		With no assumptions about the distribution of the data, the Pettitt (1979) approach of change-point detection is commonly applied to detect a single change-point in hydrological time series with continuous data
        \cite{pettittNonParametricApproachChangePoint1979}.
		It tests $H0$: The variables follow one or more distributions that have the same location parameter (no change), against the alternative: a change point exists. The non-parametric statistic is defined as:

		Mathematically, when a sequence of random variables is divided into two segments represented by $\mathrm{x}_{1}, \mathrm{x}_{1}, \ldots, x_{t_{0}} \text{and} x_{t_{0}+1}, x_{t_{0}+2}, \ldots, x_{T}$, if each segment has a common distribution function, i.e., $F_1(x)$, $F_2(x)$ and $F_1(x) \neq F_2(x)$, then the change point is identified at $t_0$. To achieve the identification of change point, a statistical index $U_{t,T}$ is defined as follows:

		\begin{equation}
			U_{t, T} = \sum_{i=1}^t\sum_{j=t+1}^T sgn(X_i - X_j), 1 \leqq t < T
		\end{equation}

		where:
		\begin{equation}
			\operatorname{sgn}(\theta)= \begin{cases}1 & \text { if } \theta>0 \\ 0 & \text { if } \theta=0 \\ -1 & \text { if } \theta<0\end{cases}
		\end{equation}

		The most probable change point $\tau$ is found where its value satisfies $K_{\tau} = max|U_{t, T}|$ and the significance probability associated with value $K_{\tau}$ is approximately evaluated as:
		\begin{equation}
			p=2 \exp \left(\frac{-6 K_{\tau}^{2}}{T^{2}+T^{3}}\right)
		\end{equation}
		Given a certain significance level $\alpha$, if $p < \alpha$, we reject the null hypothesis and conclude that $x_{\tau}$ is a significant change point at level $\alpha$.

		We used $\alpha = 0.001$ as the threshold level of p-value (see \textit{SI Appendix} Methods S5 for Sensitivity analysis), meaning that the probability of a statistically significant change-point judgment being valid was more than $99.9\%$.
		Since this method can only return one significant change point, we divided the series into two at that point and analyzed seach series separately, until all significant change points were detected.

	% % 计算贡献度
	% \subsection{Contribution decomposition}
	% 	We decomposed the amount of variation in each index at different periods to quantify each influence's contribution to the index. Using the Integrated Water Governance Index (IWGI) as an example, its value is influenced by normalized indicators in three dimensions: stress ($I'_S$), purpose ($I'_P$) and allocation ($I'_A$). We can calculate their differences between two certain years ($y_2$ and $y_1$, $y_2 > y_1$) by:
	% 	\begin{align}
	% 	\Delta IWGI &= (I'_{P_{y_2}} + I'_{A_{y_2}} - I'_{S_{y_2}}) - (I'_{P_{y_1}} + I'_{A_{y_1}} - I'_{S_{y_1}}) \\
	% 	&= (I'_{P_{y_2}} - I'_{P_{y_1}}) + (I'_{A_{y_2}} - I'_{C_{y_1}}) + (I'_{S_{y_1}} - I'_{S_{y_2}}) \\
	% 	&= \Delta I'_P + \Delta I'_A + (-\Delta I'_S)
	% 	\end{align}
	% 	Then, the contribution of dimension $x$ to IWGI's changes can be referred to as:
	% 	\begin{equation}
	% 		Contribution_x = \frac{\Delta I'_x}{\lvert \Delta IWGI \rvert}
	% 	\end{equation}

		% Since $Contribution_x$ can be positive or negative, we used an absolute value to evaluate the contribution (in proportion) of a certain dimension in the all three dimensions to the changes:
		% \begin{equation}
		% 	Contribution_x\% = \frac{|\Delta I'_x|}{\sum_x \lvert \Delta I'_x \rvert}
		% \end{equation}

		% or just for contributions (in proportion) within a certain year $j$ to the regime:
		% \begin{equation}
		% 	Contribution_{x,j}\% = \frac{\lvert I'_{x, j} \rvert}{\lvert \sum_x I'_{x, j} \rvert}
		% \end{equation}

	\subsection{Datasets}
	% 为了计算 IWGI ,我们需要计算多个指标及子指标,所有使用的数据集都在表中列出,数据的详细介绍可见补充材料。
	In order to calculate IWGI from 1965 to 2013, we need to calculate multiple indicators and sub-indicators. All the datasets used are listed in the \textit{SI Appendix} table S1. A detailed description of the data is provided in the supplementary materials \textit{SI Appendix} Methods S2.
